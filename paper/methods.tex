\section{Method}
\subsection{Subjects}
42 healthy subjects, 21 men and 21 women were recruited (age: 23.93$\pm$2.74 years, BMI: 23.66$\pm$3.28). Subjects with ongoing meditation practice, acute or chronic pain, neurological, musculoskeletal or mental illness, pregnancy or taking medications that might influence their response to pain were excluded.

\subsection{Study design}
A controlled trial was designed, whereby the subjects were assigned into a control and treatment group with an equal gender distribution, as illustrated in \autoref{fig:studydesign}.

\begin{figure}[H]
\centering
%\caption{} \vspace{-.25cm}
\includegraphics[width=1\columnwidth]{../figures/studydesign.png}
\caption{Study design}
\label{fig:studydesign}
\end{figure} 

\noindent 
The subjects of the treatment group practiced 20 minutes mindfulness meditation on 5 consecutive days between the two measurements, while the subjects of the control group continued their normal routine. The same time interval between the measurements was used for the two groups.

\subsection{Measurements}%Experimental Procedure
The testing point, as shown in \autoref{fig:trapezius}, was marked at the right upper trapezius to ensure reliable and rapid location during the experimental procedure. The location of the testing points on the upper trapezius was determined between the acromion and 7th cervical vertebra. 

Pressure Pain Threshold and Pressure Pain Tolerance were measured with an algometer (Wagner Force Ten™ Digital force Gage) three times with a 5 minutes rest period in between. The examiner was blinded during the three measurements to avoid bias. The mean of the three measurements, was computed. 

\begin{figure}[H]
\centering
\includegraphics[width=.7\columnwidth]{../figures/trapezius}
\caption{Testing point on the upper trapezius.}
\label{fig:trapezius}
\end{figure} \vspace{-.5cm}

%The treatment group practiced 20 minutes mindfulness meditation on 5 consecutive days. After the last meditation session the second measurement was conducted likewise the baseline measurement.
%The subjects of the control group continued their normal routine. The same time interval between baseline and second measurement was used for the subjects of the control group.

\subsection{Meditation Technique}
Short-term mindfulness meditation with 20 minutes of meditation on 5 consecutive days. To ensure same meditation conditions, a guided meditation in form of an audio file was used. 

A short introduction to mindfulness meditation was provided before the first meditation session. The meditation was focused attention focusing on breath flow.

\subsection{Data Analysis}
To test the effect of mindfulness meditation on chronic neck pain, a paired t-test / Mann-Whitney / Wilcoxon rank test was applied to the baseline and second measurement of both groups, the treatment and the control group.
