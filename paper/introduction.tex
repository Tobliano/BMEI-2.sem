\section{Introduction}
Approximately 20~\% of the population suffer from chronic pain \cite{Macfarlanea2016}. The characteristic of chronic pain is a duration of pain more than three months \cite{Mello2016}. Due to the persistence of pain the patients get restricted physically as well as psychically. The patients’ ability to participate in diverse activities decreases. Those activities are not only physically but also socially, maintaining an independent lifestyle and relationships to friends and family can be affected. Survey by \cite{Breivik2006} showed that pain has an impact on the work life, whereby 25\% of the patients indicated that they changed their job, responsibilities at job or lost their job due to chronic pain. Furthermore, depression was diagnosed in 21~\% of those patients. \cite{Breivik2006} One of the most common types of chronic pain is neck pain, as 25~\% suffer from this in the UK \cite{Macfarlanea2016}. Those patients are restricted by negatively affected fatigue and concentration \cite{vanRanderaat2016}. Furthermore, they suffer like the majority of chronic pain patients from anxiety and depressed mood, cognitive distress and the resulting physical limitations. \cite{Gross2013} At the moment there is no cure for chronic pain. The current treatment methods only provide possibilities to relieve the pain. \cite{Pope2017,marcus2009} Nevertheless, the majority of the patients feels pain daily and this pain is increasing throughout the day due to the daily activities. \cite{Breivik2006} 


Chronic pain is mainly treated by medication. However, medications have side effects like abuse or organ damage. To avoid those risks, alternative methods can be used. One of those methods is mindfulness meditation. Whereby meditation is used as mental training to achieve diminished judgment of emotions, cognitive control and existential insight. \cite{marcus2009} 

*** MAYBE WRITE SOMETHING ABOUT FA - MAKE IT SHORT**


Previous studies show that mindfulness meditation provides the ability to enhance a broad spectrum of cognitive health outcomes. Furthermore, stress, depression and anxiety can be relieved. These improvements are due to practicing mindfulness meditation, especially because of the mental training in emotion regulation, cognitive control, acceptance and positive mood. \cite{marcus2009,Zeidan2012}Nevertheless, there are not many studies which show the effect of mindfulness meditation on chronic neck pain. \cite{Macfarlanea2016} Additionally, the pain relieve properties are mostly investigated after practicing mindfulness meditation over a time period of two months or more. The effect of a shorter time period of mindfulness meditation on chronic neck pain is not investigated yet. 


The present study addressed if mindfulness meditation can alter pain sensation in the neck by measuring pressure pain threshold and pressure pain tolerance before and after short-term mindfulness meditation.
Therefore the hypothesis "Short-term mindfulness meditation increases the pressure pain threshold and the pressure pain tolerance in the upper trapezius" was tested.

*** CONSIDER TO WRITE SOMETHING ABOUT THE LINK BETWEEN UNHEALTHY AND HEALTHY SUBJECTS ***
