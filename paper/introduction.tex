\section{Introduction}
Approximately 20~\% of the world population suffer from chronic pain \cite{Macfarlanea2016}. The characteristic of chronic pain is a duration of pain more than three months \cite{Mello2016}. Due to the persistence of pain the patients get restricted physically as well as psychically. The patients’ ability to participate in diverse activities decreases. Those activities are not only physical but also social, maintaining an independent lifestyle and relationships to friends and family can be affected. A survey in nine European countries by Breivik et al. \cite{Breivik2006} showed that pain has an impact on the work life, whereby 25\% of the patients indicated that they changed their job, responsibilities at job or lost their job due to chronic pain. Furthermore, depression was diagnosed in 21~\% of those patients. \cite{Breivik2006} 

One of the most common types of chronic pain is neck pain, as 25~\% suffer from this in the UK \cite{Macfarlanea2016}. Those patients are restricted by negatively affected fatigue and concentration \cite{vanRanderaat2016}. Furthermore, they suffer like the majority of chronic pain patients from anxiety and depressed mood, cognitive distress and the resulting physical limitations. \cite{Gross2013} 

At the moment there is no cure for chronic pain. The current treatment methods only provide possibilities to relieve the pain. \cite{Pope2017,marcus2009} Nevertheless, the majority of the patients feels pain daily and this pain increases throughout the day due to daily activities. \cite{Breivik2006} 
Chronic pain is mainly treated by medication. However, medications have side effects like abuse or organ damage. To avoid those risks, alternative methods can be used. \cite{Pope2017,marcus2009, stein2017} One of those methods is mindfulness meditation, which uses meditation as mental training to achieve diminished judgment of emotions, cognitive control and existential insight. \cite{marcus2009} There are several types of meditation techniques. One of the most common is  focused attention (FA), which is suitable for beginners and trains the concentration by focusing on an object or specific thing, often the sensation of breath. \cite{Zeidan2016, Kabat1982}. 

Previous studies show that mindfulness meditation was able to enhance a broad spectrum of cognitive health outcomes, such as improvements of emotion regulation, cognitive control and positive mode. Furthermore stress, depression and anxiety can be relieved. These improvements are due to practicing mindfulness meditation, especially because of the mental training in emotion regulation, cognitive control, acceptance and positive mood. \cite{marcus2009,Zeidan2012, Zeidan2016} Nevertheless, there are not many studies which investigating the effect of mindfulness meditation on chronic neck pain. \cite{Macfarlanea2016} Additionally, most studies investigated the outcomes of practicing mindfulness meditation over a time period of two months or longer. The effect of a shorter time period of mindfulness meditation on chronic neck pain is not investigated yet. 


The present study address if mindfulness FA meditation can alter pain sensitivity in the neck by measuring pressure pain threshold (Threshold) and pressure pain tolerance (Tolerance). The upper trapezius is involved in chronic neck pain and this muscle present lower pressure pain threshold values compared with other muscles \cite{Fischer1987, Falla2004}, wherefore it was chosen as location for pressure application.
Therefore the hypothesis \textit{"Short-term mindfulness FA meditation increases the Threshold and Tolerance in the right upper trapezius"} was tested.

%*** Write something about why we chose the upper trapezius - look into the problem formulation in WS ***
