%----------------------------------------------------------------------------------------
%	PACKAGES AND OTHER DOCUMENT CONFIGURATIONS
%----------------------------------------------------------------------------------------

\documentclass[twoside,twocolumn]{article}

\usepackage{blindtext} % Package to generate dummy text throughout this template 

\usepackage[numbers]{natbib}
\usepackage{dblfloatfix}
\usepackage[sc]{mathpazo} % Use the Palatino font
\usepackage[T1]{fontenc} % Use 8-bit encoding that has 256 glyphs
\linespread{1.05} % Line spacing - Palatino needs more space between lines
\usepackage{microtype} % Slightly tweak font spacing for aesthetics

\usepackage[utf8]{inputenc}	
\usepackage[english]{babel} % Language hyphenation and typographical rules

\usepackage{graphicx}
\usepackage{wrapfig}
\usepackage{float}
%\usepackage{subfigure}

\usepackage[hmarginratio=1:1,top=32mm,columnsep=20pt]{geometry} % Document margins
\usepackage[hang, small,labelfont=bf,up,textfont=it,up]{caption} % Custom captions under/above floats in tables or figures
\usepackage{subcaption}
\usepackage{booktabs} % Horizontal rules in tables

\usepackage{lettrine} % The lettrine is the first enlarged letter at the beginning of the text
\usepackage{enumitem} % Customized lists
\setlist[itemize]{noitemsep} % Make itemize lists more compact

\usepackage{multirow} 

\usepackage{abstract} % Allows abstract customization
\renewcommand{\abstractnamefont}{\normalfont\bfseries} % Set the "Abstract" text to bold
\renewcommand{\abstracttextfont}{\normalfont\small\itshape} % Set the abstract itself to small italic text

\usepackage{titlesec} % Allows customization of titles
\renewcommand\thesection{\Roman{section}} % Roman numerals for the sections
\renewcommand\thesubsection{\roman{subsection}} % roman numerals for subsections
\titleformat{\section}[block]{\large\scshape\centering}{\thesection.}{1em}{} % Change the look of the section titles
\titleformat{\subsection}[block]{\large}{\thesubsection.}{1em}{} % Change the look of the section titles

\usepackage{fancyhdr} % Headers and footers
\pagestyle{fancy} % All pages have headers and footers
\fancyhead{} % Blank out the default header
\fancyfoot{} % Blank out the default footer
\fancyhead[C]{The effect of short-term mindfulness focused attention meditation on pain sensitivity} %$\bullet$ May 2018} % Custom header text
\fancyfoot[RO,LE]{\thepage} % Custom footer text

\usepackage{titling} % Customizing the title section

\usepackage[hidelinks]{hyperref} % For hyperlinks in the PDF

\renewcommand{\thetable}{\Roman{table}}
\renewcommand{\thefigure}{\Roman{figure}}


%----------------------------------------------------------------------------------------
%	TITLE SECTION
%----------------------------------------------------------------------------------------

\setlength{\droptitle}{-4\baselineskip} % Move the title up

\pretitle{\begin{center}\huge\bfseries} % Article title formatting
\posttitle{\end{center}} % Article title closing formatting
\title{The effect of short-term mindfulness focused attention meditation on pain sensitivity} % Article title
\author{%
\textsc{Annabel Bantle, Irene Uriarte Mercader, Maria Kaalund Kroustrup}\\ % Your name
\textsc{and Toby Steven Waterstone}\\[1ex]
\normalsize Aalborg University \\ % Your institution
}

\date{May 31, 2018} % Leave empty to omit a date
\renewcommand{\maketitlehookd}{%
\begin{abstract}
\noindent
\textbf{Introduction:} Studies show that long-term mindfulness meditation provides the ability to enhance a broad spectrum of cognitive health outcomes on chronic pain patients. However, there are not many studies which show the effect of short-term mindfulness on chronic neck pain. The purpose of this study was to determine if short-term mindfulness focused attention meditation can alter pain sensitivity in healthy subjects. Even though the study was conducted in healthy subjects, the effects of meditation to pain sensitivity can be transferred to chronic pain patients.
\textbf{Methods:} 42 subjects were recruited for a control trial. To evaluate the pressure pain threshold and tolerance, pressure pain was applied with an algometer on the right upper trapezius in two measurements session with 5 days in between. The treatment group practiced 20 minutes of mindfulness focused attention meditation on 5 consecutive days between the two measurements, whilst the control group continued their normal 
routine. 
\textbf{Results:} Two-way mixed ANOVA showed no significant difference between control and treatment group in threshold (P = 0.321, P = 0.507) and tolerance (P = 0.080, P = 0.472). T-test showed no significant difference in the relative difference between first and second measurement of threshold (P = 0.149) and tolerance (P = 0.330). 
\textbf{Conclusion:} Nevertheless this study still contributes to the field of pain relief using mindfulness meditation. Since this study shows the tendency that mindfulness focused attention meditation increases both, threshold and tolerance, a longer period of practicing mindfulness focused attention meditation should be investigated.
\end{abstract}
}

%----------------------------------------------------------------------------------------

\begin{document}

% Print the title
\maketitle

%----------------------------------------------------------------------------------------
%	ARTICLE CONTENTS
%----------------------------------------------------------------------------------------

\section{Introduction}
Approximately 20~\% of the population suffer from chronic pain \cite{Macfarlanea2016}. The characteristic of chronic pain is a duration of pain more than three months \cite{Mello2016}. Due to the persistence of pain the patients get restricted physically as well as psychically. The patients’ ability to participate in diverse activities decreases. Those activities are not only physically but also socially. Like maintaining an independent lifestyle and relationships to friends and family can be affected.  Pain also has an impact on the work life, where 25\% of the patients changed job, responsibilities at job or lose their job due to chronic pain. Furthermore, depression was diagnosed in 21~\% of those patients. \cite{Breivik2006}
One of the most documented types of chronic pain is neck pain where  25~\% suffer from this in the UK \cite{Macfarlanea2016}. Those patients are restricted by negatively affected fatigue and concentration \cite{Zee2016}. Furthermore, they suffer like the majority of chronic pain patients from anxiety and depressed mood, cognitive distress and the resulting physical limitations. \cite{Gross2013}
At the moment there is no cure for chronic pain patients. The current treatment methods only provide possibilities to relieve the pain. \cite{Pope2017,Dawn2009} Nevertheless, the majority of the patients feels pain daily and this pain is increasing throughout the day due to the daily activities. \cite{Breivik2006} Chronic pain is mainly treated by medication. However medications have side effects like abuse or organ damage. To avoid those risks, alternative methods are used. One of those methods is mindfulness meditation. Whereby, meditation is used as mental training to achieve diminished judgment of emotions, cognitive control and existential insight. \cite{Dawn2009}
Previous studies show that mindfulness meditation provides the ability to enhance a broad spectrum of cognitive health outcomes. Furthermore, stress, depression and anxiety can be relieved. This improvements are due to the mental training achieved by mindfulness meditation. Especially because of emotion regulation, cognitive control, acceptance and positive mood. \cite{Dawn2009,Zeidan2012} Nevertheless, there are not many studies which show the effect of mindfulness meditation on chronic neck pain. \cite{Macfarlanea2016} Additionally, the pain relieve properties are mostly investigated after practicing mindfulness meditation over a time period of two months or more.  A shorter time period of mindfulness meditation has also shown to have an effect on pain relieve, but the amount of studies investigating neck pain is limited.
The present study addressed the question if short-term mindfulness meditation can relieve neck pain by measuring pressure pain threshold and pressure pain tolerance. Therefore the hypothesis "Short-term mindfulness meditation increases the pressure pain threshold and the pressure pain tolerance in the upper trapezius" was tested.

%********This is how is writen in the report*******
% The present study addressed if mindfulness meditation can alter pain sensation in the neck by measuring pressure pain threshold and pressure pain tolerance before and after short-term mindfulness meditation.
%Therefore the hypothesis "Short-term mindfulness meditation increases the pressure pain threshold and the pressure pain tolerance in the upper trapezius" was tested.


%------------------------------------------------

\section{Method}
\subsection{Subjects}
XX healthy subjects, XX men and XX women were recruited (age: 2X$\pm$XX years, BMI: XX$\pm$XX). Subjects with ongoing meditation practice, acute or chronic pain, neurological, musculoskeletal or mental illness, pregnancy or taking medications that might influence their response to pain were excluded.

\subsection{Study design}
A controlled trial was designed, whereby the subjects were randomly assigned into a control and treatment group with an equal gender distribution between the groups, as illustrated on \autoref{fig:studydesign}.

\begin{figure}[H]
\centering
%\caption{} \vspace{-.25cm}
\includegraphics[width=1\columnwidth]{../figures/studydesign.png}
\caption{Study design}
\label{fig:studydesign}
\end{figure} \vspace{-.5cm}

The subjects of the treatment group practiced 20 minutes mindfulness meditation on 5 consecutive days between the two measurements, while the subjects of the control group continued their normal routine. The same time interval between the measurements was used for the two groups.

\subsection{Measurements}%Experimental Procedure
The testing point, as shown on \autoref{fig:trapezius}, was marked at the right upper trapezius to ensure reliable and rapid location during the experimental procedure. The location of the testing points on the upper trapezius was determined between the acromion and 7th cervical vertebra. 

For both measurements the Pressure Pain Threshold and Pressure Pain Tolerance were measured with an algometer (Wagner Force Ten™ Digital force Gage) three times with a 5 minutes rest period in between. The examiner was blinded during the three measurements to avoid bias. The mean of the three measurements, was computed. 

\begin{figure}[H]
\centering
%\caption{} \vspace{-.25cm}
\includegraphics[width=.7\columnwidth]{../figures/trapezius}
\caption{Testing points on the upper trapezius.}
\label{fig:trapezius}
\end{figure} \vspace{-.5cm}

%The treatment group practiced 20 minutes mindfulness meditation on 5 consecutive days. After the last meditation session the second measurement was conducted likewise the baseline measurement.
%The subjects of the control group continued their normal routine. The same time interval between baseline and second measurement was used for the subjects of the control group.

\subsection{Meditation Technique}
Short-term mindfulness meditation with 20 minutes of meditation on 5 consecutive days. To ensure same meditation conditions, a guided meditation in form of an audio file was used. 

A short introduction to mindfulness meditation was provided before the first meditation session. The meditation was focused attention focusing on breath flow.

\subsection{Data Analysis}
To test the effect of mindfulness meditation on chronic neck pain, a paired t-test / Mann-Whitney / Wilcoxon rank test was applied to the baseline and second measurement of both groups, the treatment and the control group.


%------------------------------------------------

\section{Results}
The tolerance for some of the subjects is not representative, as the examiner was not able to apply enough force with the algometer to reach the subjects' tolerance, thus those subjects were excluded. 

The Shapiro-Wilk test showed a normal distribution ($\alpha$ > 0.05) for the threshold and tolerance pre and post for both, treatment and control group. Therefore the two-way mixed ANOVA was applied. The test showed an equality of covariance (p = 0.955) and a equality
of error variances (p > 0.05) for threshold and tolerance pre and post. 
Whereby the pre and post measurements of threshold and tolerance were compared to see the  within-subjects effect.These results are illustrated in Table \ref{table:TWOWAYANOVA1}. The groups, treatment and control, were compared to see the between-subjects effect. These results are illustrated in  Table \ref{table:TWOWAYANOVA2}. 

\begin{table}[ht]
\caption{Results from the two-way mixed ANOVA, showing the within-subject effect. The asterisk indicates significant difference.}
\centering
\begin{tabular}{l c c}
\hline\hline
\textbf{Within-Subjects Effect} & \textbf{F} & \textbf{Sig} \\ [0.5ex] % inserts table %heading
\hline
Measurement &  13.052 &  0.001* \\
Measurement x Group & 0.154 & 0.507 \\
\hline
\end{tabular}
\label{table:TWOWAYANOVA1}
\end{table}

\begin{table}[ht]
\caption{Results from the two-way mixed ANOVA, showing the between-subject effect.}
\centering
\begin{tabular}{l c c}
\hline
\textbf{Between-Subjects Effect} & \textbf{F} & \textbf{Sig} \\ [0.5ex] % inserts table %heading
\hline
Group &  1.492 &  0.231 \\
\hline
\end{tabular}
\label{table:TWOWAYANOVA2}
\end{table}

\noindent
The test indicates that there is a significant main effect between the pre and post measurements (within-subject effect, Measurement). However, no significant main effect is seen between the treatment and control group (between-subjects effect, Group), nor a significant main interaction between the measurements and the groups (within-subjects effect, Measurement$^{x}$Group).

The Shapiro-Wilk test showed a normal distribution ($\alpha$ > 0.05) for the difference in threshold and tolerance as percentage of the pre and post for both, treatment and control group. Therefore the T-test was applied. There is an equality of error variances for the tolerance (p = 0.159) and no equality of error variance for threshold (p = 0.013). The results from the T-test are illustrated in Table \ref{table:TTEST}. 

\begin{table}[ht]
\caption{Results from the T-test.}
\centering
\begin{tabular}{c c}
\hline\hline
\textbf{Threshold} & \textbf{Tolerance} \\ [0.5ex] % inserts table %heading
\hline
 0.149 &  0.330 \\
\hline
\end{tabular}
\label{table:TTEST}
\end{table}

The test indicates that there is no significant difference in the threshold and tolerance.

%------------------------------------------------

\section{Discussion}

\subsection{Summary and interpretation of the findings}
There was seen an overall increase in the threshold and tolerance within the two measurements for both, the treatment and control group. However, no significant improvement of the pressure pain threshold and pressure pain tolerance between the groups was found. Furthermore, no significant difference between the difference as a percentage in threshold and tolerance was found. But a tendency can be seen that the treatment group has a higher percentage increase in both threshold and tolerance compared with the control group. 

\subsection{Experimental Setup}
%More standardization of the experiment in some way… We rely on a person putting pressure on the subjects (Maybe own section: XXX)
One of the drawbacks of the manual algometer is the difficulty in assessing objectively the rate in pressure applied. Different studies insist in the importance of training and practice with the algometer in order to achieve reliable values. However, due to the thigh time to execute the project, an appropriate training period was not possible, which would be convenient.

%Subjects: “ It’s hard to define the first sensation of pain.”
Different subjects expressed difficulties ratting their own threshold, why it is challenging to find true values of pain. Hence, this research rely on the ability of the subjects to rate their pain.

%What if we only look at the threshold and not the tolerance. If you are a trained person you may have a higher tolerance compared to other subjects. 
It appears more convenient to focus on the pressure pain threshold instead of the pressure pain tolerance, because of the big variability in tolerance measures and validity of the measures obtained from the subjects, because pain tolerance is harder to reach with the algometer.

%Would it have any influence if the subject should be sore in the muscle before the measurement due to training? (Yes, so an exclusion criteria would be not to train the muscle for maybe 2 days before the experiment)
A study by Tesarz et al. \cite{Tesarz2012} concludes that pain perception can be altered by physical activity. Subjects with good physical condition participating in the study, showed higher threshold and tolerance values compared with other subjects. Nevertheless, this fact does not affect the outcomes of the study because we compared the subjects with themselves, not with the others.
Along a study by Koltyn et al. \cite{Koltyn2002} concludes that high-intensity exercise is followed by hypoalgesia, which leads to an increased pain threshold as well as pain tolerance values during and after exercise. Based on this, the exclusion criteria should take into account that subjects cannot train before the measurement.

\subsection{Meditation technique}
Other studies have shown that mindfulness meditation has an effect on pain. Those studies investigated the effect of a meditation practice over two months or more using MBSR. \cite{Kabat1982,Rosenzweig2010} The effect on pain intensity and pain unpleasantness of short-term mindfulness meditation practice was shown by Zeidan et al. \cite{Zeidan2012}. However, Zeidan et al. \cite{Zeidan2012} used a meditation technique which was a combination of FA and OM, particularly focusing on pain-related brain processing. Whereas this study was investigating the effect of regular short-term mindfulness FA meditation. Hence one could speculate that different meditation types affect pain after various time periods of practice and that 5 consecutive days are not sufficient to elicit mindfulness FA meditation’s modulation of pain.

Nevertheless, there were some limitations within the used meditation technique. Potentially the used audio-guide did not ensure that the subjects understood the principles of mindfulness FA meditation, even though an introduction to mindfulness meditation was given orally on the first day. However, this introduction was provided by a non-specialist, who possibly did not know the key focus of explaining mindfulness meditation to laymen. This uncertainty was based on board spectrum of mindfulness meditation techniques and their unclear delineations. 
Furthermore, the subjects were told to meditate in the most comfortable position, which varied from subject to subject. These inconsistent sitting positions may have influenced the meditation outcome of single subjects. In addition, there was no control, if the subjects were meditating in the right way.


%------------------------------------------------

\section{Conclusion}
Short-term mindfulness FA meditation on 5 consecutive days did not show a significant effect on pressure pain sensitivity in the right upper trapezius. However, a significant effect was found between Pre and Post measurement for treatment and control group, which was seen as an increase in Threshold and Tolerance. This increase between Pre and Post measurement indicates the habituation effect to pressure pain.
On the behalf of this a clear conclusion about the effect of short-term mindfulness FA meditation on pain sensitivity cannot be stated. Nevertheless this study provides insight into pain relief using mindfulness meditation.
Since this study shows the tendency that 5 consecutive days of mindfulness FA meditation practice increase Threshold and Tolerance, a longer period should be investigated. 
Furthermore, a comparison of this and
other studies indicates that the effect of mindfulness meditation varies depending on the
meditation technique. Hence the effects of the different meditation techniques should be
further investigated in order to evaluate if different meditation techniques provide various
effects on pain relief.



%----------------------------------------------------------------------------------------
%	REFERENCE LIST
%----------------------------------------------------------------------------------------


\footnotesize{\bibliographystyle{unsrt}
\bibliography{C:/Users/tobbe/Documents/GitHub/BMEI-2.sem/setup/BMEI_2.sem}


%\endgroup

%----------------------------------------------------------------------------------------

\end{document}
