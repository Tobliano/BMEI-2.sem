\section{Discussion}

\subsection{Summary and interpretation of the findings}
%There was seen an overall increase in the threshold and tolerance within the two measurements for both, the treatment and control group. However, no significant improvement of the pressure pain threshold and pressure pain tolerance between the groups was found. Furthermore, no significant difference between the difference as a percentage in threshold and tolerance was found. But a tendency can be seen that the treatment group has a higher percentage increase in both threshold and tolerance compared with the control group. 

A significant difference is found between the measurements, Pre and Post, indicated by the two-way mixed ANOVA. However, no significant difference in pressure pain threshold and pressure pain tolerance between the groups is found. Furthermore, no significant difference in Improvement in threshold and tolerance is found between the groups, indicated by the t-test. Nevertheless a tendency can be seen that the treatment group has a higher Improvement in both, threshold and tolerance, compared with the control group. These results indicate that there might be a habituation effect on pressure pain. 

\subsection{Experimental Setup}
%More standardization of the experiment in some way… We rely on a person putting pressure on the subjects (Maybe own section: XXX)
%One of the drawbacks of the manual algometer is the difficulty in assessing objectively the rate in pressure applied. Different studies insist in the importance of training and practice with the algometer. However, due to the avaliable time to execute the project, an appropriate training period was not possible, which would be convenient in order to achieve reliable values.

One of the drawbacks of the manual algometer is the difficulty in assessing objectively rate in pressure application. %, it is difficult to increase the pressure uniformly. 
%The examiner in charge of the experiment is a strong male, however he had difficulty to apply enough force to reach the pressure pain tolerance for some subjects. Some factors could affect this outcome like inappropriate technique using the algometer, examiner's fatigue after several measurements or the standing position during the experiment. 
According to different studies insist in the importance of training and practice with the algometer  *** Missing citation ***. However, due to the available time to execute the project, an appropriate training period was not possible, which would be convenient in order to achieve  more reliable values.
%Subjects: “ It’s hard to define the first sensation of pain.”
%Different subjects expressed difficulties ratting their own threshold, why it is challenging to find true values of pain. Hence, this research rely on the ability of the subjects to rate their pain.

%What if we only look at the threshold and not the tolerance. If you are a trained person you may have a higher tolerance compared to other subjects. 
%It appears more convenient to focus on the pressure pain threshold instead of the pressure pain tolerance, because of the big variability in tolerance measures and validity of the measures obtained from the subjects, because pain tolerance is harder to reach with the algometer.



%Pain tolerance is less used for research purposes due to ethical reasons as well as its high variability among the subjects \cite{Yarnitsky2006}. 
It appears convenient to only focus on the pressure pain threshold instead of the pressure pain tolerance. This is not only because of the extensive variety in the results, but also the validity of the measurements, as for some subjects it was not possible to reach a representative pressure pain tolerance.
%Would it have any influence if the subject should be sore in the muscle before the measurement due to training? (Yes, so an exclusion criteria would be not to train the muscle for maybe 2 days before the experiment)
%A study by Tesarz et al. \cite{Tesarz2012} concludes that pain perception can be altered by physical activity. Subjects with good physical condition participating in the study, showed higher threshold and tolerance values compared with other subjects. Nevertheless, this fact does not affect the outcomes of the study because we compared the subjects with themselves, not with the others.
%Along a study by Koltyn et al. \cite{Koltyn2002} concludes that high-intensity exercise is followed by hypoalgesia, which leads to an increased pain threshold as well as pain tolerance values during and after exercise. Based on this, the exclusion criteria should take into account that subjects cannot train before the measurement.

A study by Tesarz et al. \cite{Tesarz2012} concludes that pain perception can be altered by physical activity. Subjects with good physical condition participating in the study showed higher threshold and tolerance values compared with other subjects. %The muscle of these subjects is also more appropriate to apply the pressure on.
%Nevertheless, this fact does not affect the outcomes of the study because we compared the subjects with themselves, not with the others. 
Furthermore a study by Koltyn et al. \cite{Koltyn2002} determines that high-intensity exercise is followed by hypoalgesia. Therefore pain threshold and tolerance values increase during and after exercise. The exclusion criteria should take into account that subjects cannot practice physical exercise involving the upper part of the thorax before the measurements.
\subsection{Meditation technique}
%Other studies have shown that mindfulness meditation has an effect on pain. Those studies investigated the effect of a meditation practice over two months or more using MBSR. \cite{Kabat1982,Rosenzweig2010} The effect on pain intensity and pain unpleasantness of short-term mindfulness meditation practice was shown by Zeidan et al. \cite{Zeidan2012}. However, Zeidan et al. \cite{Zeidan2012} used a meditation technique which was a combination of FA and OM, particularly focusing on pain-related brain processing. Whereas this study was investigating the effect of regular short-term mindfulness FA meditation. Hence one could speculate that different meditation types affect pain after various time periods of practice and that 5 consecutive days are not sufficient to elicit mindfulness FA meditation’s modulation of pain.

%Nevertheless, there were some limitations within the used meditation technique. Potentially the used audio-guide did not ensure that the subjects understood the principles of mindfulness FA meditation, even though an introduction to mindfulness meditation was given orally on the first day. However, this introduction was provided by a non-specialist, who possibly did not know the key focus of explaining mindfulness meditation to laymen. This uncertainty was based on board spectrum of mindfulness meditation techniques and their unclear delineations. 
%Furthermore, the subjects were told to meditate in the most comfortable position, which varied from subject to subject. These inconsistent sitting positions may have influenced the meditation outcome of single subjects. In addition, there was no control, if the subjects were meditating in the right way.
There were some limitations within the used meditation technique. Potentially the used audio-guide did not ensure that the subjects understood the principles of mindfulness FA meditation, even though an oral introduction was given on the first day. However, this introduction was provided by a non-specialist, who possibly did not know the key focus of explaining mindfulness FA meditation to laymen. This uncertainty was based on the board spectrum of mindfulness meditation techniques and their unclear delineations. 

Furthermore, the subjects were told to meditate in the most comfortable position, which varied between the subjects. Inconsistent sitting positions may have influenced the meditation outcome of single subjects. In addition, there was no control, if the subjects were meditating adequately.

Other studies have shown that mindfulness meditation has an effect on pain. Those studies investigated the effect of meditation practice over two months or more using mindfulness-based stress reduction. \cite{Kabat1982,Rosenzweig2010} The effect on pain intensity and pain unpleasantness of short-term mindfulness meditation practice was shown by Zeidan et al. \cite{Zeidan2012}. However, Zeidan et al. \cite{Zeidan2012} used a meditation technique which combine FA and open monitoring, particularly focusing on pain-related brain processing. Whereas this study was investigating the effect of short-term mindfulness FA meditation. Hence pain relief is affected not only by the type of meditation but also by the practice period. Therefore 5 consecutive days may not be sufficient to elicit mindfulness FA meditation’s modulation of pain.