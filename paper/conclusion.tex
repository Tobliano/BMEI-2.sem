\section{Conclusion}
Short-term mindfulness FA meditation on 5 consecutive days did not show a significant effect on pressure pain sensitivity in the right upper trapezius. However, a significant effect was found between Pre and Post measurement for treatment and control group, which was seen as an increase in Threshold and Tolerance. This increase between Pre and Post measurement indicates the habituation effect to pressure pain.
On the behalf of this a clear conclusion about the effect of short-term mindfulness FA meditation on pain sensitivity cannot be stated. Nevertheless this study provides insight into pain relief using mindfulness meditation.
Since this study shows the tendency that 5 consecutive days of mindfulness FA meditation practice increase Threshold and Tolerance, a longer period should be investigated. 
Furthermore, a comparison of this and
other studies indicates that the effect of mindfulness meditation varies depending on the
meditation technique. Hence the effects of the different meditation techniques should be
further investigated in order to evaluate if different meditation techniques provide various
effects on pain relief.

