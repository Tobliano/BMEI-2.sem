\section{Conclusion}

Short-term mindfulness FA meditation on 5 consecutive days did not show a significant effect on pressure pain relief in the right upper trapezius. However, a significant effect was found between Pre and Post measurement for treatment and control group, which was seen as an increase in Threshold and Tolerance. This increase between Pre and Post measurement  indicates the habituation effect to pressure pain.
A clear conclusion about the effect of mindfulness FA meditation on pain relief cannot be stated. Nevertheless this study still contributes to the field of pain relief using mindfulness meditation as an alternative method. Since this study shows the tendency that 5 consecutive days of mindfulness FA meditation practice increase Threshold and Tolerance, a longer period should be investigated. Furthermore, this study indicates that the effects of mindfulness meditation varies depending on the meditation technique. Hence the effects of the different meditation techniques should be further investigated in order to evaluate if different meditation techniques provide various effects on pain relief.

