\section{Results}
The tolerance for some of the subjects is not representative, as the examiner was not able to apply enough force with the algometer to reach the subjects' tolerance, thus those subjects were excluded. Therefore,
the results are based on 32 subjects, 15 subjects in the treatment and 17 subjects in the control group. 

The Shapiro-Wilk test showed a normal distribution ($\alpha$ > 0.05) and the Levene's test showed equal variance ($\alpha$ > 0.05) for the threshold and tolerance pre and post for both, treatment and control group. Therefore the two-way mixed ANOVA was applied. Whereby the pre and post measurements for both the threshold and tolerance were compared to see the within-subjects effect and the groups, treatment and control, were compared to see the between-subjects effect. The results from the threshold pre and post are illustrated in Table \ref{table:TWOWAYANOVA1}.   The results from the tolerance pre and post are illustrated in Table \ref{table:TWOWAYANOVA2}. 

\begin{table}[ht]
\caption{Results from the two-way mixed ANOVA for threshold. The asterisk indicates significant difference.}
\centering
\begin{tabular}{l c c c}
\hline \hline
\textbf{Within-Subjects Effect} \\
& \textbf{df} &\textbf{F} & \textbf{Sig} \\ [0.5ex] % inserts table %heading
\hline
Measurement & 1 & 13.052 &  0.001* \\
Measurement x Group & 1 & 0.451 & 0.507 \\
\hline \hline
\textbf{Between-Subjects Effect}  \\
& \textbf{df} & \textbf{F} & \textbf{Sig} \\ [0.5ex] % inserts table %heading
\hline
Group & 30 & 1.492 &  0.231 \\
\hline
\end{tabular}
\label{table:TWOWAYANOVA1}
\end{table}

\noindent
The test indicates that there is a significant main effect between the pre and post for the threshold measurements (within-subject effect, Measurement), F(1,30) = 13.051, p=0.001. However, no significant main effect is seen between the treatment and control group for threshold (between-subjects effect, Group), F(1,30)=1.492, p = 0.231 nor a significant main interaction between the treshold measurements and the groups(within-subjects effect, Measurement x Group), F(1,30)=0.451, p=0.507.

\begin{table}[ht]
\caption{Results from the two-way mixed ANOVA for tolerance. The asterisk indicates significant difference.}
\centering
\begin{tabular}{l c c c}
\hline \hline
\textbf{Within-Subjects Effect} \\ 
& \textbf{df} & \textbf{F} & \textbf{Sig} \\ [0.5ex] % inserts table %heading
\hline
Measurement & 1 &  8.918 &  0.006* \\
Measurement x Group & 1 & 0.532 & 0.472 \\
\hline \hline
\textbf{Between-Subjects Effect} \\ 
 & \textbf{df} & \textbf{F} & \textbf{Sig} \\ [0.5ex] % inserts table %heading
\hline
Group & 30 & 3.289 &  0.080 \\
\hline
\end{tabular}
\label{table:TWOWAYANOVA2}
\end{table}

\noindent
The test indicates that there is a significant main effect between the pre and post for the tolerance measurements (within-subject effect, Measurement), F(1,30) = 8.981, p=0.006. However, no significant main effect is seen between the treatment and control group for threshold (between-subjects effect, Group), F(1,30)=3.289, p = 0.080 nor a significant main interaction between the treshold measurements and the groups(within-subjects effect, Measurement x Group), F(1,30)=0.532, p=0.472. 

The Shapiro-Wilk test showed a normal distribution ($\alpha$ > 0.05) and an unequal variance (p = 0.013)for the improvement in threshold and tolerance for both, treatment and control group. Therefore the T-test was applied. The results from the T-test are illustrated in Table \ref{table:TTEST}. 

\begin{table}[ht]
\caption{Results from the T-test.}
\centering
\begin{tabular}{c c}
\hline\hline
\textbf{Threshold} & \textbf{Tolerance} \\ [0.5ex] % inserts table %heading
\hline
 0.149 &  0.330 \\
\hline
\end{tabular}
\label{table:TTEST}
\end{table}

\noindent
The test indicates that there is no significant difference in the improvement in threshold and tolerance between the two groups.