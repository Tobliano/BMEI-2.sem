\section{Summary and interpretation of the findings}
There was seen an overall increase in the threshold and tolerance within the two measurements for both, the treatment and control group. However, no significant improvement of the pressure pain threshold and pressure pain tolerance between the groups was found. Furthermore, no significant difference between the difference as a percentage in threshold and tolerance was found. But a tendency can be seen that the treatment group has a higher percentage increase in both threshold and tolerance compared with the control group. 

%All subjects: (Threshold Treatment: 38.35 \% $\pm$ 61.11, Tolerance Treatment: 25.47 \% $\pm$ 33.22, Threshold Control: 26.42 \% $\pm$ 41.68, Tolerance Control: 12.77 \% $\pm$ 27.27). 
%Not all subjects: (Threshold Treatment: 51.32 \% $\pm$ 67.06 , Tolerance Treatment:  21.50 \% $\pm$ 31.12, Threshold Control: 22.68 \% $\pm$ 33.19, Tolerance Control: 12.00 \% $\pm$ 23.02 ). 

\section{Experimental Setup}
%More standardization of the experiment in some way… We rely on a person putting pressure on the subjects (Maybe own section: XXX)
Among the limitations of the study is the algometer. One of the drawbacks of the manual algometer is the difficulty in assessing objectively the rate in pressure applied. For the examiner it is difficult to increase the pressure gradually. Even though the examiner in charge of the experiment is a well-trained male, he found it tough to apply enough pressure to reach the pressure pain tolerance for some subjects. Different studies insist in the importance of training and practice with the algometer in order to achieve reliable values. However, due to the thigh time to execute the project, an appropriate training period was not possible, which would be convenient.


%Discuss the algometer. Would it be better to choose another measurement method? Cold/heat - cuff?
A study by Neddermeyer et al. \cite{Neddermeyer2007} found out that different kind of stimuli \fxnote{hot, cold, electric current, blunt pressure and punctate pressure} measure a common pathophysiological process implicit in nociception. Therefore the pain threshold value of a person does not depend on the stimulus. Based on these findings the pressure algometer is as reliable as any other type of stimuli application.
%The standing position for the examiner could be better, maybe we could have raised the bench
%If measurements were close after each other, the examiner was not able to apply the same pressure in each measurement. (Jorge Michael got weaker/exhausted) → Just use the pressure pain threshold instead of taking pressure pain tolerance into account (most studies does this, only using the threshold).
%What should we do about these subjects where we were not able to apply the pressure to? 

%Subjects: “ It’s hard to define the first sensation of pain.”
The values for the pressure pain threshold and pressure pain tolerance from the subjects collected during the experiment were based on their personal sensation of pain. However, different subjects expressed difficulties ratting their own threshold. It is known that pain is a subjective matter, thus it is challenging to find true values of pain. Hence, this research rely on the ability of the subjects to rate their pain.

%What if we only look at the threshold and not the tolerance. If you are a trained person you may have a higher tolerance compared to other subjects. 
Pain tolerance is less used for research purposes due to not only  ethical reasons but also its high variability among the subjects \cite{Yarnitsky2006}. Pain threshold values seem to be constant, however pain tolerance values are altered by psychological and psychosocial factors. Because of this extensive variety in the results, it appears convenient to focus on the pressure pain threshold instead of the pressure pain tolerance. 

%Would it have any influence if the subject should be sore in the muscle before the measurement due to training? (Yes, so an exclusion criteria would be not to train the muscle for maybe 2 days before the experiment)
Different studies have investigated the effect of exercise in pain perception. It has been found that pain thresholds as well as pain tolerances increased during and after exercise.  A study by Koltyn et al. \cite{Koltyn2002} concludes that high-intensity exercise is followed by hypoalgesia. Based on this, the exclusion criteria should take into account that subjects cannot train before the measurements.

%Our choice of muscle - would it be better with another muscle with easier accesses/not as soft and which might have lower pain threshold and tolerance? 
%Does it have any influence if the muscle is more trained for some subjects than others?  (No, this would be the same for both measurements  and we are only comparing the subject with itself)
There are significant differences in the pressure pain threshold values depending on the  muscle under study. It was found in a study by Fischer et al. \cite{Fischer1987} that muscles located in the lower part of the body present  higher pressure thresholds. Accordingly, the upper trapezius showed the lowest pressure pain threshold values for both male and females compared with other muscles.

A study by Tesarz et al. \cite{Tesarz2012} concludes that pain perception can be altered by physical activity. Subjects with good physical condition participating in the study, showed higher threshold and tolerance values compared with other subjects. Nevertheless, this fact does not affect the outcomes of the study because we compared the subjects with themselves, not with the others.

\section{Meditation technique}
Other studies have shown that mindfulness meditation has an effect on pain. Those studies investigated the effect of a meditation practice over two months or more using MBSR. \cite{Kabat1982,Rosenzweig2010} The effect on pain intensity and pain unpleasantness of short-term mindfulness meditation practice was shown by Zeidan et al. \cite{Zeidan2012}. However, Zeidan et al. \cite{Zeidan2012} used a meditation technique which was a combination of FA and OM, particularly focusing on pain-related brain processing. Whereas this study was investigating the effect of regular short-term mindfulness FA meditation. Hence one could speculate that different meditation types affect pain after various time periods of practice and that 5 consecutive days are not sufficient to elicit mindfulness FA meditation’s modulation of pain.

Nevertheless, there were some limitations within the used meditation technique. Potentially the used audio-guide did not ensure that the subjects understood the principles of mindfulness FA meditation, even though an introduction to mindfulness meditation was given orally on the first day. However, this introduction was provided by a non-specialist, who possibly did not know the key focus of explaining mindfulness meditation to laymen. This uncertainty was based on board spectrum of mindfulness meditation techniques and their unclear delineations. 
Furthermore, the subjects were told to meditate in the most comfortable position, which varied from subject to subject. These inconsistent sitting positions may have influenced the meditation outcome of single subjects. In addition, there was no control, if the subjects were meditating in the right way.
