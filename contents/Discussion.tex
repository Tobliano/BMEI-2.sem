\section{Summary and interpretation of the findings}
There was seen an overall increase in the threshold and tolerance within the two measurements for both, the treatment and control group. However, no significant difference in pressure pain threshold and pressure pain tolerance between the groups was found. Through a significant difference was found between the measurements, Pre and Post, indicated by the two-way mixed ANOVA. Furthermore, no significant difference in Improvement in threshold and tolerance was found between the groups, indicated by the t-test. Nevertheless a tendency can be seen that the treatment group has a higher Improvement increase in both, threshold and tolerance, compared with the control group. These results indicate that there might be a habituation effect on pressure pain. 
A study by Bingel et al. \cite{Bingel2007} showed, in healthy subjects, that pain habituates over time when they had to rate their pain \cite{Bingel2007}. 

\section{Experimental Setup}
%More standardization of the experiment in some way… We rely on a person putting pressure on the subjects (Maybe own section: XXX)
%Among the limitations of the study is the algometer. 
One of the drawbacks of the manual algometer is the difficulty in assessing objectively the rate in pressure application, it is difficult to increase the pressure uniformly. The examiner in charge of the experiment is a strong male, however he had difficulty to apply enough force to reach the pressure pain tolerance for some subjects. Some factors could affect this outcome like inappropriate technique using the algometer, examiner's fatigue after several measurements or the standing position during the experiment. Accordingly different studies insist in the importance of training and practice with the algometer. However, due to the available time to execute the project, an appropriate training period was not possible, which would be convenient in order to achieve  more reliable values.

There are significant differences in the pressure pain threshold values depending on the investigated muscle. A study by Fischer et al. \cite{Fischer1987} find out that muscles located in the lower part of the body present higher pressure thresholds. Accordingly, the upper trapezius showed the lowest pressure pain threshold values for both males and females compared with other muscles. Therefore the upper trapezius is suitable to measure pressure pain threshold because the necessary amount of force is not as high as in other muscles. The low threshold values measured in this muscle allow to observe changes between the two measurement sessions trouble-free. Furthermore a survey by Neddermeyer et al. \cite{Neddermeyer2007} found out that the pain threshold of a person does not depend on the stimulus source \fxnote{hot, cold, electric current, blunt pressure and punctate pressure}. Based on these findings the algometer is as reliable as any other type of stimuli application.

**** ADD SOMETHING ABOUT TOLERANCE HERE ***
%A study by Neddermeyer et al. \cite{Neddermeyer2007} found out that different kind of stimuli \fxnote{hot, cold, electric current, blunt pressure and punctate pressure} measure a common pathophysiological process implicit in nociception. Therefore the pain threshold value of a person does not depend on the stimulus. Based on these findings the pressure algometer is as reliable as any other type of stimuli application.



%***** Maybe used this for argument for why we chose 3 measurements **************** HERE! HERE!
%Different subjects expressed difficulties ratting the threshold values even though were based on their personal sensation of pain. Pain is a subjective matter, thus it is challenging to find true values of pain. The validity of the results can be compromise if the subjects do not fully understand how to rate the first sensation of unpleasantness. It can be difficult for the participants to have a clear distinction between discomfort and pain. Consequently it is not possible to correctly indentify the true threshold values. Hence, this research rely on the ability of the subjects to rate their own pain. 


Pain tolerance is less used for research purposes due to not only  ethical reasons but also its high variability among the subjects \cite{Yarnitsky2006}. Pain threshold values seem to be relatively less variable, however pain tolerance values are altered by psychological and psychosocial factors. It appears convenient to only focus on the pressure pain threshold instead of the pressure pain tolerance. This is not only for the extensive variety in the results, but also the validity of the measurements as it was for some subjects not possible to reach a representative pressure pain tolerance.


%Would it have any influence if the subject should be sore in the muscle before the measurement due to training? (Yes, so an exclusion criteria would be not to train the muscle for maybe 2 days before the experiment)
%Different studies have investigated the effect of exercise in pain perception. 

A study by Tesarz et al. \cite{Tesarz2012} concludes that pain perception can be altered by physical activity. Subjects with good physical condition participating in the study, showed higher threshold and tolerance values compared with other subjects. The muscle of these subjects is also more appropriate to apply the pressure on.
%Nevertheless, this fact does not affect the outcomes of the study because we compared the subjects with themselves, not with the others. 
Along a study by Koltyn et al. \cite{Koltyn2002} determines that high-intensity exercise is followed by hypoalgesia. Therefore pain threshold as well as pain tolerance values increase during and after exercise. The exclusion criteria should take into account that subjects cannot practice physical exercise involving the upper part of the thorax before the measurements.

%Our choice of muscle - would it be better with another muscle with easier accesses/not as soft and which might have lower pain threshold and tolerance? 
%Does it have any influence if the muscle is more trained for some subjects than others?  (No, this would be the same for both measurements  and we are only comparing the subject with itself)



%*** Write the part above more together with the part above this part 'Different studies have investigated....' - Bo comment: But maybe the subject had physical exercises before one of the measurements, which could affect the results ***

\section{Meditation technique}
There were some limitations within the used meditation technique. Potentially the used audio-guide did not ensure that the subjects understood the principles of mindfulness FA meditation, even though an introduction to mindfulness meditation was given orally on the first day. However, this introduction was provided by a non-specialist, who possibly did not know the key focus of explaining mindfulness meditation to laymen. This uncertainty was based on board spectrum of mindfulness meditation techniques and their unclear delineations. 
Furthermore, the subjects were told to meditate in the most comfortable position, which varied from subject to subject.  Inconsistent sitting positions may have influenced the meditation outcome of single subjects. In addition, there was no control, if the subjects were meditating in the adequate.

Other studies have shown that mindfulness meditation has an effect on pain. Those studies investigated the effect of a meditation practice over two months or more using MBSR. \cite{Kabat1982,Rosenzweig2010} The effect on pain intensity and pain unpleasantness of short-term mindfulness meditation practice was shown by Zeidan et al. \cite{Zeidan2012}. However, Zeidan et al. \cite{Zeidan2012} used a meditation technique which was a combination of FA and OM, particularly focusing on pain-related brain processing. Whereas this study was investigating the effect of regular short-term mindfulness FA meditation. Hence pain relief is affected not only by the type of meditation but also by the practice period depending on the meditation technique. Therefore 5 consecutive days may not sufficient to elicit mindfulness FA meditation’s modulation of pain.

%Hence one could speculate that different meditation types affect pain after various time periods of practice and that 




