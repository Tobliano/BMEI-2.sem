\chapter{Focused Attention Meditation}
\textbf{Title:} Focused Attention Meditation $\sim$  Sensations of Breathing  \\
\textbf{Length of the whole tape:} 22:40 \\
\textbf{Meditation time:} 20 minutes and 1.5 minutes of introduction \\
\textbf{Background music:} Buddhist meditations music for positive energy  \\
\textbf{Speaker:} David Noyce 

During the whole recording calm music is playing in the background. From time to time the guidance through the meditation is provided by a male voice, which is illustrated in the text below in italics. 

00:01 - 01:27: \textit{Hello and welcome, this is a guided focused attention meditation in which will be focusing our attention on the sensations we can feel around our natural breathing process. Our breathing is something we can always feel natural happening, and focusing on it can help us to be fully present and to take a break from focusing on thinking. I will say a couple of things about this meditation, and then you will hear a bell ring three times to begin. Sometimes, as we are trying to focus on the sensation of breathing we might find our mind getting distracted, and this is a total natural part of meditation, it does not mean that we are doing it wrong, all we can do in those moments is notice that we have been come distracted and without judging our selves, remember our intention to focus on the sensation of breathing, no matter how many times our attention wonders, we learn to just notice and gently bring it back to the sensation of breathing, it might help to keep our focus somewhere in the body where we can feel the sensation of breathing clearly. So now let focus on our attention on the natural sensation of breathing. It might help us to start out with a few deep slow breaths, filling our lungs with air, and breathing out worries attention.}

01:28 - 01:36: Only background music 

01:36 - 01:51: Three times sound of a gong 

01:51 - 02:30: Only background music

02:31 - 02:46: \textit{Whenever we notice our attention has wondered away from the breath we can just take a moment to be present with whatever our experience is, and then we can gently bring our focus back to the natural sensation of breathing. }

02:47 - 06:34: Only background music

06:34 - 06:42 \textit{If you have notice that your attention is no longer present with the sensation of breathing, just take a moment to notice where it is, and then gently bring it back. }

06:43 - 11:35: Only background music

11:35 - 11:38 \textit{Lets keep focusing on the sensation of our natural breathing.}

11:39 - 16:37: Only background music

16:38 - 16:45: \textit{At some point we might be able to follow the whole cycle of sensations, all the way through the in-breath and all the way through the out-breath. }

16:46 - 19:35: Only background music

19:36 - 19:41: \textit{We will meditate for another two minutes, so remember our attention to focus on the sensation of breathing.}

19:42 - 21:35: Only background music 

21:35 - 21:45: Two times sound of a gong  

21:45 - 22:40: Only background music 