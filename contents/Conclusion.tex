Short-term mindfulness FA meditation on 5 consecutive days did not show an significant effect on pressure pain relief on the upper trapezius, in this study, when comparing the treatment group to the control group. Only a significant effect was found from pre and post measurement, shown as an increase of pressure pain, but this was also the case for the control group, thus in a lesser degree. Wherefore a clear conclusion on the effect of mindfulness meditation on pain relief cannot be stated. Nevertheless this study still contributes to the field of pain relief using mindfulness meditation as an alternative method. Since this study shows the tendency that mindfulness FA meditation increases both, threshold and tolerance, a longer period of practicing mindfulness FA meditation should be investigated. Furthermore, this study indicates that the effects of mindfulness meditation varies depending on the meditation technique. Hence the effects of the different meditation techniques should be further investigated in order to evaluate if different meditation techniques have various effects on pain relief. 