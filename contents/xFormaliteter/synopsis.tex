Studier har påvist, at langvarig mindfulness meditation kan forbedre en række kognitive funktioner for patienter med kroniske smerter. 
Imidlertid er antallet af studier, der undersøger effekten af kortvarig mindfulness meditation for patienter med kroniske nakkesmerter begrænset. 

Formålet med dette studie er, at påvise om kortvarig mindfulness focused attention meditation kan påvirke smerte sensitivitet. Tryk smerte var påført på den højre øvre trapezius på raske forsøgspersoner med et algometer over to måling sessioner med 5 dages mellemrum. Herved blev smertetærsklen og smertetolerancen evalueret. Behandlingsgruppen udøvede 20 minutter mindfulness focused attention meditation over 5 sammenhængende dage, mens kontrolgruppen fortsatte deres normale rutiner mellem målingerne.

Resultater viser ingen signifikant forskel mellem behandlings- og kontrolgruppen. Yderligere er ingen signifikant ændring påvist i smertetærskelen og smertetolerancen. På trods af dette bidrager studiet stadig til området inden for smertelindring ved brug af mindfulness meditation. 

Yderligere forskning er dog nødvendig for at undersøge effekten af mindfulness focused attention meditation over en længere periode eftersom, dette studie viser en tendens til at smertetærskelen og smertetolerancen øges.
