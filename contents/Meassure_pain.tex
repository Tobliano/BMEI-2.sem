\section{How to measure pain}
The International Association for the Study of Pain (IASP) defines pain as an unpleasant sensory and emotional experience associated with actual or potential tissue damage, or described in terms of such damage\cite{(Marsky and Bogduk, 1994}. Pain is described as a complex and subjective experience that poses a number of measurement challenges due to its subjective nature. Nevertheless, pain measurements are necessary for pain studies as well as the evaluation of methods to control pain.\cite{libro pain}
There is no valid and reliable method of objectively quantifying pain at the moment. However, despite the challenges that pain measurement present, several tools and approaches can be employed in order to collect useful pain estimates. \cite{pain outcomes paper} 

 %\subsection{Statistical and Clinical Significance}
 %A treatment should demonstrate a statistically as well as a clinically significant effect. 
 %"P value problem". To overcome this problem interpretations should rely on the effect size. The effect size is expressed in clinical terms, as the reduction of pain scores.

 \subsection{The Qualities and Components of Pain}
 Pain can be classify attending at the site of the origin of the pain in \textit{somatic pain} and \textit{visceral pain}. Somatic pain occurs when pain receptors or \textit{nociceptors} in tissues (skin, muscles, skeleton, joints, and connective tissues) are activated. This kind of pain is usually localized to a specific area. Visceral pain, is defined as pain that results from the activation of nociceptors of the thoracic, pelvic, or abdominal viscera. Unlike somatic pain,is harder to localize within the body. Both types compose what is called \textit{nociceptive pain}. On the other hand the IASP defines \textit{neuropathic pain} as pain initiated or caused by a primary lesion or dysfunction of the peripheral or Central Nervous System (CNS). The main difference between nociceptive and neuropathic pain is the absence of a continuous nociceptive input in the latest \cite{neuropathic pain}. The most common presentations of this condition are \textit{hyperalgesia}  and \textit{allodynia}. Neuropathic pain can be sub
There is another classification of pain attending to the duration of the same. \textit{Acute pain} is the pain associated with tissue damage, this pain is usually limited to the site of the damage and it is possible to locate it accurately. The magnitude of the pain is dependent on the intensity of the stimulus. However, there are cases when the pain persist for long times or return repeatedly, this forms of pain are called \textit{chronic pain}. Neuropathic pain is a variety of chronic pain.

 \subsubsection{Adaptation to pain}
 \subsection{Pain assessment}
The aim of pain assessment is to diagnose the cause, understand the impact, identify appropriate pain relief strategies and evaluate their effectiveness.\cite{art and science}. There are different dimensions of pain experience that can be assessed: pain intensity, pain affect, pain quality and pain location. Due to the fact that pain is a multidimensional experience, it is needed a multidimensional assessment. %The principal areas are location, intensity, quality, onset and duration, previous treatments, associated symptoms and effect on activities. 
%There are different purposes on regards of assessment of pain and are categorized as clinical outcomes, epidemiological and quality improvement. 
Because of the fact that pain is a subjective, internal and personal event which cannot be register directly by clinicians, it is frequently use patient self-report indices to asses the experiences of pain \cite{libro pain}. Within this category it is possible to find unidimensional as well as multidimensional scales.
 \subsection{Unidimensional scales}
 %Maybe make subdivisions i
Unidimensional scales explore only one dimension of pain. The most common assessed dimension of pain is its intensity. This could be due to the fact that patients are usually able to provide quantitative pain intensity relatively rapidly \cite{libro pain}. One commonly used unidimensional tool are the Verbal Rating Scales (VRS) which consists of a list of adjectives describing different levels of pain intensity. It is important for an accurate measure using this scale to provide adjectives which reflect the extremes of the dimension, as well as additional adjectives to represent the different gradations of pain. Patients are asked to select the word that best describes their level of pain intensity. VRSs are usually scored by listing the adjectives in order of pain severity and assigning each one score as a function of its rank. This type of scales are easy to administer, score and apprehend. However, it has several statistical disadvantages and criticism raised due to the fact that assumes equal intervals between adjectives \cite{libro pain}. For this particular reasons along with others is used when the patient conditions require it \cite{six methods paper}. Other possibility of unidimensional scales is a visual analogue scale (VAS). A VAS consist of a 10 cm line the ends of this line are labeled as the extremes of pain. Patients are asked to indicate the point along the line that best represents the intensity of their pain. The scale is scored by measuring the distance from 'no pain' end to the patient's mark. They are usually measure in millimeters thus, for a 10 cm line gives a high number of response categories. This fact makes the VAS more sensitive to changes in pain intensity. However, one of the drawbacks is that scoring time is higher than for other methods. Numerical Rating Scale (NRS) is also within unidimensional tools of pain intensity measure. A NRS consist of asking the patient to rate his or her perceived level of pain intensity on a numerical scale from 0 to 10 (or 0 to 100), being described 0 as 'no pain' and  10 or 100 equal to 'higest level of pain'. The advantage of  NRS is that not require patients mobility because the response is given verbally. NRS are valid and demonstrate positive and significant correlations with other measures of pain intensity \cite{six methods paper}. Pictures or Face Scales illustrate facial expressions of different intensities of pain. Even though the primary purpose of this scales was to offer individuals with written language or cognitive difficulties an option to express pain intensity, there is evidence that they are valid \cite{libro pain}. 
 \subsection{Multidimensional scales}
 Multidimensional scales are convenient in relentless pain conditions. Multidimensional scales measure several dimensions of pain with different combinations of these dimensions. They offer a more detailed reflection of the patient's pain experience \cite{art and science}. Within this category we can find the McGill Pain Questionnaire (MPQ), which consist of 78 word that describe pain in sensory, affective and evaluative terms. The words have been arranged in group each one of them with similar sensory qualities and ranked according to their intensity. A six point VRS is also included. It is possible to obtain a numerical score of the components of pain adding a score to the ranked words. The MPQ is been proved as a valid method support by several studies \cite{libro pain}. A brief form of this questionnaire it has been introduced, the short-form McGill Pain Questionnaire (SF-MPQ). The patients are able to rate the pain with 15 different descriptors in sensory and affective terms. Each descriptor is rated on a 4-point scale. The SF-MPQ includes a VAS for pain intensity as well as a VRS for rating the overall pain experience. Even though the brief pain inventory (BPI) was developed to assess cancer pain has been proven as a useful instrument to asses different kinds of pain in several clinical settings \cite{libro pain}. The BPI measures pain severity, pain quality and the disturbance caused in the patient' 'daily life'. Two subscale scores pain intensity and pain interference.  

%The West Haven-Yale Multidimensional Pain Inventory (WHYMPI) is a comprehensive pain outcomes measure that contains 52 items and 12 subscales. Some of the subscales include perceived interference of pain in a variety of areas, response from significant others, pain severity, perceived life control, affect, and participation in various work, social, and personal activities. Items are assessed on a 7-point scale. The scale can also yield clinically useful types of pain patients, such as dysfunctional, interpersonally depressed, and adaptive copers
\begin{itemize}

\item treatment outcomes of pain survey
\end{itemize}
\subsection{Behavioral Measurements}
It is suggested that if a patient has pain, visible signs of discomfort, behavioral and/or physiological will be present. Even so, absence of this signs does not mean absence of pain. For this reason behavioral scales should be used in non-communicative patients (cognitively impaired adults and children), when self-report assessment of pain is not possible and not instead of self-report scales.\cite{referncia buscar}