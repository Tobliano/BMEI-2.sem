\subsection{Psychophysical methods}
Quantitative sensory testing (QST) evaluates the integrity of the entire sensory neuraxis receptor to the cortex. Even though QST has recieved criticism for being subjective, it is a reliable test. Brain imaging studies provided evidence that subjective pain magnitude scores are associated with objectively measured neural activity in areas of the brain involved in pain processing. QST include different modalities of stimulation, such as thermal, mechanical, electrical, ischemic and chemical. This method provide two different assessments of pain. On the one hand the  evaluation of endogenous pain, which is the pain that the patient experiences due to the disease process. On the other hand, the assessment of induced pain, in order to experiment on pain mechanisms or therapy. \cite{neurop_exam}

%Psycophysics: investigates quantitative the relationship between physical stimuli and the sensations and perceptions they affect.
%Psycophysical methods provide measures for perception and performance. Two approaches thresholds and scaling.

\subsubsection{Measurement of experimental pain}
 As a result to a set of experimental noxious stimuli it is possible to obtain different parameters such as, pain thresholds, tolerance or suprathreshold pain intensities. Threshold is defined as the stimulus that produces an arbitrary, but defined, level of performance. There is a distinction between receptor or absolute threshold and psychophysical or sensory threshold. Absolute threshold is the energy required to elicit response in the primary afferent while the psychophysical or sensory threshold, is the minimal energy necessary to reach perception. Due to the fact that receptor threshold is lower than sensory threshold, the sensory threshold is a convenient parameter which offers the transition point between non-painful and painful stimulus \cite{neurop_exam}.
\begin{itemize}
	
	\item Methods of adjustment. 
	\item Methods of limits
	In this method, different magnitude stimuli are presented to the patient in ascending or descending order. The subject indicates whether or not the stimulus are detected. Acordingly the threshold in each case is the stimulus magnitude at which the response switches from non perception to perception and/or vice versa \cite{chapter3}.
	
	A potential disadvantage of the method of limits is that the observer may become accus tomed to reporting that they perceive (or not) a stimulus, and as a result continue to give the same response even at stimulus magnitudes that are higher (or lower) than the “real” threshold. This is termed the error of habituation. Conversely, the observer may anticipate that the stimulus is about to become detectable, or undetectable, and make a premature judgment. This is called the error of expectation
	
	One of the drawbacks using this method is that the parameter value from perception to non-perception differ from non-perception to perception value, due to different artifacts contributing to this different.
	\item Method of constant stimuli
	\item Two-alternative forced choice (2AFC)
\end{itemize}