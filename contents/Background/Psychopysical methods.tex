\subsection{Psychophysical methods}
Quantitative sensory testing(QST) evaluates the integrity of the entire sensory neuraxis receptor to the cortex and it is a reliable and relatively reproducible test. Even though QST has been criticized for being subjective, brain imaging studies, provide evidence that subjective pain magnitude scores are associated with objectively measured neural activity in areas of the brain involved in pain processing.
Psycophysics: investigates quantitative the relationship between physical stimuli and the sensations and perceptions they affect.
Psycophysical methods provide measures for perception and performance. Two approaches thresholds and scaling.

\subsubsection{Thresholds}
Threshold: the stimulus that produces an arbitrary, but defined, level of performance.
\begin{itemize}
	\item Method of constant stimuli
	
	\item Methods of limits
	This method has been used to measure not only absolute threshold but also difference thresholds. In order to apply this method, different magnitude stimuli are presented to the patient in ascending or descending order. The subject indicates whether or not the stimulus is detected
	One of the drawbacks using this method is that the parameter value from perception to non-perception differ from non-perception to perception value, due to different artifacts contributing to this different.
	\item Methods of adjustment
	\item Two-alternative forced choice (2AFC)
\end{itemize}