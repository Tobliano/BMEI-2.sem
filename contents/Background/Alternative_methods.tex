\section{Alternative methods for pain treatment}
Alternative methods for treating pain alternative to medication like analgesia. Some alternative methods will be presented in the following sections. 

\subsection{Chiropractor}
Adjustment and manipulation of the spinal cord to reduce pressure on the nerves running down the spinal cord.
In a study by ... evaluating 506 patients with acute and chronic back pain after 3 month of chiropractic treatment.
Patients undergoing chiropractic treatment show improvements in their condition, the effect is ongoing after 3 month.  
\subsection{Acupuncture}
Acupuncture is a treatment where small sterile needles are inserted into the skin of the patient. The needles are inserted at specific acupuncture points related to the type of pain that the patient is experiencing. 
In a study by \cite{Junnilla1983} acupuncture has shown promising results in reducing pain in patients with soft tissue round the shoulder joint, headaches, neck and shoulder pain, arthritis/osteoarthritis and low vack pain. A total of 348 patients where evaluated. The mean reduction of the entire patient group where 68 \%. Showing best results in soft tissue round the shoulder joint, showin a mean reduction of pain by 79 \%. The headache and neck and shoulder patients had a mean reduction by 74 \%. Patients with  arthritis/osteoarthritis showed a mean reduction by 58 \% and the patients with low back pain had a mean reduction by 50 \%. In 80 \% of the patients the effect of the treatment lasted for more than 3 month and 32 \% over one year. \cite{Junnilla1983}

\subsection{Yoga}
Yoga is a form of mind to body practice discipline, or tradition originating from India. In the practice of yoga different physical postures, breathing techniques and more are the routine. 
Yoga is both a form of personal evolution, but most popular because of the exercise which benefits the health.
A review by \cite{...} found that yoga could improve the functionality of the back and a slight effect of treating pain compared to non-yoga participants. 
%\subsection{\}Exercise} 
%\subsection{Hypnosis}

\subsection{Mindfulness meditation}
Enhanced emotion regulation, cognitive control, acceptance and positive mood have been linked with health benefits as well as pain modulation. These mechanisms has been shown to be modulated during mindfulness meditation practice.
A study by Perlman et al. (...) shows that practicing meditation could not lower the intensity of pain, but instead lower pain unpleasentness in the participants. The findings on meditation and pain modulation are split, but experiments in controlled settings are still needed to confirm if the effect of mindfulness meditation works on pain modulation. \cite{Zeidan2012}

When using a placebo analgesia the typical response is increased activation of the dorselateral prefrontal cortex during anticipation of pain. Effect that predicts reductions in pain perception and activaty of pain related brain regions. Mindfullness does not involve DLPFC activation. 