\subsection{Psychophysical methods}
Quantitative sensory testing (QST) evaluates the integrity of the entire sensory neuraxis receptor to the cortex. Even though QST has recieved criticism for being subjective, it is a reliable test. Brain imaging studies provided evidence that subjective pain magnitude scores are associated with objectively measured neural activity in areas of the brain involved in pain processing. QST include different modalities of stimulation, such as thermal, mechanical, electrical, ischemic and chemical. This method provide two different assessments of pain. On the one hand the  evaluation of endogenous pain, which is the pain that the patient experiences due to the disease process. On the other hand, the assessment of induced pain, in order to experiment on pain mechanisms or therapy. \cite{neurop_exam}

%Psycophysics: investigates quantitative the relationship between physical stimuli and the sensations and perceptions they affect.
%Psycophysical methods provide measures for perception and performance. Two approaches thresholds and scaling.

\subsubsection{Measurement of experimental pain}
 As a result to a set of experimental noxious stimuli, it is possible to obtain different parameters such as, pain thresholds, tolerance or suprathreshold pain intensities. Threshold is defined as the stimulus that produces an arbitrary, but defined, level of performance. There is a distinction between receptor or absolute threshold and psychophysical or sensory threshold. Absolute threshold is the energy required to elicit response in the primary afferent while the psychophysical or sensory threshold, is the minimal energy necessary to reach perception. Due to the fact that receptor threshold is lower than sensory threshold, the sensory threshold is a convenient parameter which offers the transition point between non-painful and painful stimulus \cite{neurop_exam}.
 
 \subparagraph{Psychophysical Procedure}
 Psychophysical research has been mostly concentrated on thresholds measurement owing to, the desire to isolate low-level sensory mechanisms using operationally defined tasks that are intended to minimize the roles of perception and cognition \cite{psy_methods}. There are different procedures in order to measure thresholds.
 
 Performance-Based Procedures 
 Appearance-Based Procedures
 
\begin{itemize}
	
	\item Methods of adjustment 
	The test subject adjust the magnitude of a stimulus, until a prespecified criterion is reached. This method is commonly used for appearance-based tasks. Currently, this method is not commonly used to obtain performance measures, due to the fact that forced-choice procedures are consider superior. However, the method of adjustment is useful for obtaining a rough threshold estimate to guide the choice of stimulus magnitudes for a forced-choice procedure, when there are different conditions to be measured.
	\item Methods of limits. In this method, different magnitude stimuli are presented to the test subject, in ascending or descending order. The subject indicates whether or not the stimulus are detected on each presentation. Accordingly, the threshold in each case is the stimulus magnitude at which the response switches from non perception to perception and/or vice versa. The patient's response cannot be evaluated if it is correct or incorrect. \cite{chapter3}. One of the drawbacks of this method  is the observer may get used to reporting that is perceiving a stimulus or not. As a result, he or she  continues to give the same response even at stimulus magnitudes that are higher or lower than the threshold. This phenomena is the error of habituation. Contrarily, the observer may anticipate the response and make a premature judgment, which is call the error of expectation. Another disadvantages using this method is that the parameter value from perception to non-perception differ from non-perception to perception value, due to different artifacts \cite{hysteresis}.
	\item Method of constant stimuli. The stimulus magnitude on each trial is randomly selected from a predefined set. This range is selected to straddle the threshold value. This method generates data, when this data fitted with the appropriate psychometric function, provides the most accurate estimates of the threshold. The choice of this stimulus set sometimes demand pilot work to obtain an estimate of the threshold. The method of adjustment, as explained before, can be useful for this purpose. It is possible to use this method simultaneously with appearance-based procedures. The selection of the stimuli range is crucial. In order to avoid the problem of selecting an incorrect set, an adaptive or staircase procedure is apply.
	\item Adaptive or staircase procedure. An algorithm, that analyzes the previous trials response, selects the stimulus magnitude on each trial. This method can be used simultaneously  with conventional methods as well as with performance-based and appearance-based tasks.
\end{itemize}


Forced-choice Performance Procedures. Forced-choice tasks can be termed by Alternative Forced-Choice (AFC) or Interval Forced-Choice (IFC). In IFC procedures the stimulus are presented in temporal order. There are different varieties within forced-choice performance procedures, 2AFC procedures are the most popular in psychophysics. In this method, in each trial two stimuli are presented. One of this stimui is the target, which the test subject must select.