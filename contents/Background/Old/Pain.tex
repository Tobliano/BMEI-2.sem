\section{Pain}
The definition of pain is an unpleasant sensory and emotional experience associated with actual or potential tissue damage~\cite{Steeds2013, Mello2016}. Pain can be categories as acute, chronic and intermittent pain or a combination based on the pain experience~\cite{Goldberg2011}. 

Pain is a sudden or slow onset of any intensity from mild to severe pain~\cite{Mello2016}. Acute pain is anticipated or predictable \cite{Mello2016}. Contrary,  chronic pain is not anticipated or predictable and has a constant or recurring of pain with a duration greater than three months~\cite{Mello2016}.


\subsection{Prevalence and incidence}
Pain is a worldwide problem, as it has a high prevalence and incidence despite the complexity of quantifying pain\cite{Goldberg2011}. It is estimated that 20\% of the world's populations adults suffer from pain and each year is 10 \% of adults diagnosed with chronic pain~\cite{Goldberg2011}. Pain affects all populations regardless of gender, age, income, ethnicity or geography, but the distribution across the globe differs\cite{Goldberg2011}. 

The frequently causes of pain are operations, cancer, osteoand rheumatoid arthritis, injuries and spinal cord problems \fxnote{make the etiology of pain a complex transdisciplinary affair}~\cite{Goldberg2011}. Furthermore, can pain causes to different sequelae, such as depression, inability to work, limit social relationships and suicidal thoughts\cite{Goldberg2011}. 

People with chronic pain often complain of cognitive problems which interfere with their daily functions~\cite{Geisser2018}. Additionally, it is indicated that among people with chronic pain there is a consistent evidence for disturbances in attentional capacity, processing speed, and psychomotor speed~\cite{Geisser2018}. However, is the relationship between pain and cognitive problems unknown~\cite{Geisser2018}. 

*********** MAYBE WRITE SOME MORE ABOUT HOW IT AFFECT THEIR DAILY LIVING? ****

\subsection{Types of pain}
Pain represents a warning sign and is difficult to measure as the experience is subjective~\cite{Steeds2013} and is affected by previous experience~\cite{Ahmad2014}. The experience of pain is multidimensional, where the sensory discriminate aspect involves intensity, quality, and location of the pain~\cite{Ahmad2014}. The cognitive and emotional factors compose of subjective psychological variables, as attention, anxiety, fear, expectation, and anticipation~\cite{Ahmad2014}. Pain can be divided into nociceptor pain and neuropathic pain~\cite{Steeds2013}

\subsection{Nociceptor pain}
Nociceptors, also known as tonic receptors or afferent nerve fiber\fxnote{tonic receptors: a sensory receptor that continues to trigger a response for minutes or hours after it is stimulated.}, are nerve cells which initiate the sensation of pain~\cite{Steeds2013, Martini2012}. Nociceptors are free nerve endings and have a high threshold for mechanical, chemical or thermal stimulation~\cite{Steeds2013}. There are two types of nociceptors $\alpha\delta$ and C fibers. $\alpha\delta$ fibers are very small, between 2-5$\mu$m, myelinated nerve cells, which produce fast well localized sharp pain~\cite{Steeds2013}. The distribution of these is in body surface, muscles and joints. C fibers are small, <2$\mu$m, unmyelinated nerve cells, which produce slow and poorly localized pain~\cite{Steeds2013}. These produce a burning or throbbing pain and the distribution of these is in most tissues~\cite{Steeds2013}. 

\subsubsection{Pain pathway}


*** MAYBE PICTURE OF PATHWAYS ****

Nociceptors are activated when a noxious stimulation occur and chemicals will be released by immune cells~\cite{Martini2012}. These are the factors that stimulate the nociceptors~\cite{Martini2012}. When the nociceptors are stimulated they will propagate the pain information to the spinal cord nearby~\cite{Martini2012}. The nociceptors will bring this pain information into the spinal cord from the dorsal horn and release neurotransmitter~\cite{Martini2012}. A second neuron, the second order neuron, will receive this information and cross over to the opposite side of the spinal cord~\cite{Martini2012}. The second order neuron brings the information towards the brain via the lateral spinothalamic tract~\cite{Martini2012}. This information will be transmitted by releasing neurotransmitters to the third order neuron in the thalamus~\cite{Martini2012}. The third order neuron is location and discrimination the pain~\cite{Martini2012}. The location of where the pain has occurred correlates to the area of the somatosensory cortex, as illustrated on \figref{fig:somatosensorycortex}. Pain on the right side of the body is processed on the left side of the brain and vice versa~\cite{Martini2012}. 

*** MAYBE PICTURE OF somatosensory cortex ****

Pain is modulation by the descending pathways, where the Periaqueductal Grey (PAG) and the Nucleus Raphe Magnus (NRM) are involved in reducing pain~\cite{Steeds2013}. PAG, also known as anti-nociceptor, is important in the control of pain and surrounds the cerebral aqueduct in Mesencephalon~\cite{Steeds2013}. When this region is electrical stimulated it produces profound analgesia and injection of morphine, this can happen when PAG receives an input from the thalamus, hypothalamus, cortex and the correlates from the spinothalamic tract~\cite{Steeds2013}. Neurons from the PAG region excite the cells in NRM which have a direction towards the spinal cord and to block pain transmission by the dorsal horn cells~\cite{Steeds2013}. Stimulation of NRM produce a strong analgesia and release serotonin which activates the inhibitory interneuron and blocks the pain transmission~\cite{Steeds2013}. The key neurotransmitter is noradrenaline and 5-hydroxytryptamine by modulation pain~\cite{Steeds2013}. 

\subsection{Neuropathic pain}
Neuropathic pain is caused by a disorder in the somatosensory system and is often a chronic condition related to injuries or diseases~\cite{Mindruta2013}. The disease occurs at different levels in the nervous system and affects the signaling of pain~\cite{Mindruta2013}. The neuropathic pain would not be described as a single cause or a single specific lesson, but instead, they would be described based on a mechanism~\cite{Mindruta2013}. This mechanism can, however, produce painful symptoms in the same disease, by it would take different aspects~\cite{Mindruta2013}.
The sensation is described as a sudden pain which is burning, tingling, shooting stabbing or numb and can be paroxysmal or continuous. The pain can be divided by the evoked pain into hyperalgesia, hyperpathia, hyperaesthesia, allodynia, and dysaesthesia. Hyperalgesia is the pain of abnormal severity followed by a noxious stimulation. Hyperpathia is an exaggerated and prolonged response to stimulation, which can be delayed in onset and after repeated stimulation. Hyperaesthesia is defined as an increased sensitivity to stimulation. Allodynia is a painful response to a normally innocuous stimulus. Dysaesthesia is an evoked or spontaneous altered sensation is described as a discomfort rather than pain. 
It can be difficult to localized the distribution of pain because the distribution is longer respect by nerves, roots, segments, proximal or distal territories as for nociception pain~\cite{Mindruta2013}. 
Neuropathic pain syndromes can be categorized into peripheral, central or mixed~\cite{Mindruta2013}. Mixed pain syndromes could example be a chronic low back pain with radiculopathy\fxnote{where one or more nerves do not work properly}. Therefore is a detailed analysis of the somatosensory abnormalities in the given case is important to distinguish neuropathic pain from nociception~\cite{Mindruta2013}. An ancillary test can be used to diagnose if the pain is caused by an injury in the somatosensory system~\cite{Mindruta2013}. This test divided information on the basis of fibers, sensation, clinical and laboratory tests~\cite{Mindruta2013}. 