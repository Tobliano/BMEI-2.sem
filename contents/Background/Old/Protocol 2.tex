\section{Protocol}

\subsection{Hypothesis}
\begin{itemize}
	\item Pain can be positively affected by mindfulness.
	\item  Pain, short-term meditation, mindfulness, pain threshold increased,
	\item Literature show a relieve of pain after meditation. Mindfulness meditation after brief training reduces pain.(Zeidan)
	\item Does short-term mindfulness meditation affect the pressure pain threshold (and pain tolerance)?
\end{itemize}

Hypothesis: Short-term mindfulness meditation increases the pressure pain threshold and the pressure pain tolerance.
\subsection{Purpose}
\subsection{Subjects}
Twenty (more better)healthy subjects were recruited for the experiment (x male, y female,mean age=Z).(We want same amount of male and female if it is possible). Specific inclusion and exclusion criteria have been formed for this experiment. It is not necessary that subjects believe in the effect of mindfulness meditation. \\

\textbf{Inclusion criteria}

\begin{itemize}
	\item Healthy subjects age between 20-30 years
	\item Must have time to meditate for 4-5 days, 20 minutes per day.
	\item Normal BMI (F: 19-24 M: 20-25)
\end{itemize}

\textbf{Exclusion criteria}

\begin{itemize}
	\item Ongoing meditation practice 
	\item Acute or chronic pain
	\item Pregnancy
	\item Neurological, musculoskeletal or mental illness
	\item Lack of ability to cooperate
	\item Signs or symptoms of any serious systemic diseases 
	\item Psychiatric, analgesic or other medications that might influence the response to pain 
	\item Abusive drug or alcohol use
	
\end{itemize}
\subsection{Setup}
The Pressure Pain Threshold (PPT), defined as the pressure at which the sensation changed from pressure to pain, has been recognized as an effective and reliable way to quantify pain measures. In this study PPT was measured using (our ALGOMETER). PPT were measured in (point of the wrist). Testing points were marked to ensure reliable and rapid location during the experimental procedure. 
(The algometer was applied/ The examiner perform the measures) three times and the average of the registrations was filed. The subjects had a (5-10min??) resting time between measurements. PPT values were measured two times, the first day of the study and after 4-5 days since the first measure.
\subsection{Approach}
For this particular experiment a parallel study was conducted. The subjects recruited for the experiment were randomly assigned in two different groups, the control group or the treatment group.(equal number of male and female?). The control group consisted of X subjects no meditation. The treatment group consisted of Y subjects meditation

\subsection{Procedure}

Control group
Treatment group
