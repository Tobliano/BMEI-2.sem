\section{Mindfulness Meditation}
Mindfulness is usually defined as being in the mental state of non-elaborative, non-judgmental awareness \cite{Zeidan2012,Zeidan2016,Tang2017}. Mindfulness is viewed upon as a lifestyle and this kind of lifestyle can be practiced through meditation, which is called mindfulness meditation. Practicing mindfulness meditation includes control over sensory, emotional and cognitive happenings. Hereby the ability to control these sensations without being distracted by them as so the ability to abstract from past and future representations of memory. Thus can be said that mindfulness meditation is training of the mind. \cite{Tang2017}
Thoughs and emotions are involve in the preception of pain.
Mindfulness meditation will not make the pain go away, but the patient will be able to deal with it easier ,reducing the fear associated with pain. Thereby the subjects engage more in the treatment instead of reeling and focusing on the medication. \cite{Jacob2016}

\subsection{Meditation classification}
The most well practiced types of meditation are focused attention (FA) and open monitoring (OM).\cite{Zeidan2016}\\
\textbf{Focused attention}\\ 
FA is the training of concentration, the subjects keep their focus at an object or specific thing, only focusing on that thing. Often the flow of breath is the focus, when practicing FA meditation.  When any disturbance comes by, like a thought, sound or other environmental distractions, which will often lead to a drift in attention, the person should always bring his or her attention back to the focus. \cite{Zeidan2016}\\
%This kind of meditation has shown to enhance focus and concentration. ..
\textbf{Open monitoring}\\
OM is the cultivation of open presence, were the mind is open to anything, not focusing on any specific thing, just being in the present. If any thought or disturbance comes by, the thought or sensation should be noticed briefly, but then left without thinking more over it. It is believed that this form of meditation is easier to learn when the person masters the meditation of FA, whereby the OM form is easier to master. \cite{Zeidan2016}

%This kind of meditation has been shown to reduce pain more compared to FA, likely because the areas of the brain affected during this form of meditation is \cite{Perlman2010}

%FA and OM can alter pain in different ways...
%OM is more effective in reducing pain after extensive meditation training compared to FA. \cite{Varilly2012}

%(.....MAYBE WE NEED SOME KIND OF REASONING TO WHY WE CHOOSE TO LOOK INTO MINDFULNESS MEDITATION, LIKE A SUMMARY OF ALL THE METHODS BEFORE GOING INTO MINDFULNESS MEDITATION, AND THEN DIG DEEPER INTO THE FACTS OF MINDFULNESS MEDITATION....)

\subsection{Mechanisms of mindfulness meditation}

Enhanced emotion regulation, cognitive control, acceptance and positive mood have been linked with health benefits as well as pain modulation. These mechanisms have been shown to be modulated during mindfulness meditation practice. A study by Perlman et al. \cite{Perlman2010} shows that practicing meditation could not lower the intensity of pain, but instead lower pain unpleasantness in the participants. \cite{Zeidan2012, Perlman2010}

The typical response, when using a placebo analgesia is, increased activation of the dorselateral prefrontal cortex during pain anticipation. Effect that predicts reductions in pain perception and activity of pain related brain regions. Mindfulness meditation does not involve dorselateral prefrontal cortex activation. \cite{Zeidan2012}

The findings on mindfulness meditation and pain modulation are split, but experiments in controlled settings are still needed to confirm if the effect of mindfulness meditation works on pain modulation. \cite{Zeidan2012, Perlman2010}

%\fxnote{....and then we need something like: or maybe a summary of all the methods and then saying, because of this and this, mindfulness meditation will be looked further upon as method for reliving pain...}

%The method of mindfulness meditation will be used for method to relief pain in this project.

%\section{Mindfulness meditation}
%Definition
Different brain regions are involved the practice of mindfulness meditation. The most important are the prefrontal cortex (PFC), involving the anterior cingulate cortex (ACC) and the medial PFC as illustrated in \figref{fig:brain_meditation}. The striatum, the insula and the default mode network (DMN), which include the medial PFC and the posterior cingulate cortex (PCC). These regions play a big role in the effect of mindfulness meditation and are highly regulating the mechanisms of meditation which can generally be catergorized into three catagories: attention control, emotion regulation and self-awareness. 

%Figure \ref{fig:brain_meditation} shows an image of the brain and the regions involved in attention, emotion and self awareness. 

\begin{figure}[H]
	\includegraphics[width=1\textwidth]{figures/brain_meditation.png} 
	\caption{Image of the brain highlighting specific regions relevant when practicing meditation}
	\label{fig:brain_meditation}  
	\cite{Tang2017}
\end{figure}   

\textbf{Attention control}\\
Attention control is the ability to maintain focus, for instance on the breath during FA meditation. This mechanism includes mainly ACC, PFC and the striatum. Increased activity in the dorsal lateral PFC is required to hold an increased attention, as well as deactivation of the areas of the brain that makes the mind drift, which include the medial PFC. \cite{Tang2017}.

\textbf{Emotion regulation}\\
Emotion regulation include the emotions that arise, when they occur and how they are experienced and expressed. This mechanism involves multiple prefrontal regions, limbic regions and striatum, which are regions primary in regulating the emotional thoughts through the limbic system also responsible for goal setting. This need for regulating the emotional control is important because during the meditation practice the participant need to be able to handle boredom or negative mood during the meditation. Stronger subgenual and adjacent ventral ACC activity with meditation. This brain area is involved with emotion regulation and attention control. The dorsal lateral PFC and amygdala plays some role in regulation of emotion. 
%What is this?

\textbf{Self-awareness}\\
Self-awareness includes the awareness of one self, the awareness of being conscious as well as meta-awareness which is the awareness of the internal bodily state. Regions of the brain involves midline cortical structure DMN, ACC, the insula, medial PFC and PCC. Reduce activity in midline cortical structure including the DMN, more reduction in the posterior part PCC, than the anterior part medial PFC, but increase in perigenual ACC activity.

\subsection{Meditation practice}
Different expertises of meditation, early, middle, advanced appear to modulate the dynamic balance between anterior and posterior midline networks involved in different aspects of self, cognitive self, bodily self, and phenomenal experiential self. This reflects self plasticity following meditation. 
The effort to get into the meditative state takes varies according to your experience level with meditation. Often this experience level can be divided into three stages, early, middle and advanced practice of meditation. These stages, illustrated in figure \ref{fig:meditation_stages}, determine the amount of effort to get into the meditative state\cite{Tang2017}. 

\begin{figure}[H]
	\includegraphics[width=0.8\textwidth]{figures/stages_of_meditation.png} 
	\caption{The three stages of meditation practice, describing how much effort one must use to get into the meditative state}
	\label{fig:meditation_stages}  
	\cite{Tang2017}
\end{figure}  

In the early stages more mental effort is required, here the dorsal lateral PFC and partial cortex are often involved and activated more. A stronger deactivation in the DMN is shown to occur when using more effort. With less effort, the ACC and striatum will participate more. \cite{Tang2017}

The neural mechanisms behind mindfulness meditation in reliving pain has been researched. Experiments where stimulating with nociceptive pain there has been shown an increase in activity areas of the PFC when meditating. Participants express that they are able to feel the pain but able to deal with it better during meditation focusing on the breath. 
The mechanisms working in analgesia are not the same as the mechanisms during meditation, why the two methods don't interfere with each other. \cite{Jacob2016}

The different areas of the brain show either a reduction or increase in activity when performing meditation. Through meditation the person trains the mind, and specific regions will grow. \cite{Zeidan2012}

Examining long term meditators, the findings are a thicker gray matter in mid cingulate cortex and bilateral secondary somatosensory cortex, which are involved in pain related regions overlapping the functional effect. A correlation with the number of years practicing meditaiton and the mid cingulate was also found. This gives evidence to long lasting effects of meditation. \cite{Zeidan2012}

However, short-term mindfulness training can have positive effect in pain relieve. The study by \cite{Zeidan2012} showed an effect of mindfulness meditation practice during four days for 20 min per session, even though most studies conduct the experiments for a period of more than six weeks. \cite{Zeidan2012}