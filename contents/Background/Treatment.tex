%\section{Symptoms of chronic pain}

%Chronic pain can occur in the whole body. Typical complains of chronic pain patients, regardless of the localization of the pain, is insidious onset of pain, non-localized pain or back pain that radiates to the leg. \cite{marcus2009}

\section{Treatment of chronic pain} \label{sec:treatment}
There are several treatments for chronic pain patients, depending on the modalities and intensity of the pain. Besides conservative methods, alternative methods are applied to reduce chronic pain. %The benefit of the alternative methods is a treatment without the risk of negative side-effects. 
\cite{marcus2009,pope2017}

None of the different treatment methods are enough or sufficient when applied alone. But an individual combination considering the needs of each patient alleviates the suffering of the chronic pain. At the moment it is not possible to cure chronic pain, but to relieve the suffering. \cite{marcus2009,pope2017} 

The commonly used treatment method is medication. The disadvantage of medication is the risk of side effects. In contrast to medication, alternative methods do not provide any negative side effect. However some of those methods require a specialist for instruction and/or application, which results in high cost. \cite{marcus2009,pope2017}
%A lot of different methods for relieving pain exist, some of these will be described in the following subsections. 

\subsection{Medication}
Medication is a common way to treat severe chronic pain patients. %, although there are disadvantages. 
Those medicaments can be divided in three groups, the coanalgesic medicaments, the non-opioid and the opioid analgesics. \cite{marcus2009}

\begin{itemize}
\item \textbf{Coanalgesics} are normally used to treat other diseases, for example depressions, but still provide analgesic qualities. They are often used to treat fibromyalgia, chronic headache and neuropathic pain. Often coanalgesics are combined with analgesics to extended pain-relief. \cite{marcus2009}

\item \textbf{Non-opioid analgesics} are used to reduce intermittent mild to moderate pain. To this category belong nonsteroidal anti-inflammatory drugs, which decrease inflammation and provide analgesic properties. Non-opioid analgesics are especially used in short-term-therapy. Non-opioid analgesics inhibit the prostaglandin synthesis. Prostaglandin has a protective effect to the organs. A permanent use of non-opioid analgesics encourages prostaglandin effects, which conduct in severe organ toxicity. Known side effects are for example gastrointestinal toxicity, nephrotoxicity and an increased risk of cardiovascular diseases. \cite{marcus2009,stein2007}

\item \textbf{Opioid analgesics} provide stronger analgesic qualities than non-opioid analgesics and show no prostaglandin effect. These analgesics work by bending in the central nervous system to the opioid or NMDA receptors.  Because of this better long-term tolerability, opioid analgesics are used in patients which suffer from chronic non-malignant pain. But the use of opioid analgesics accompanies with the risk of abuse and misuse. Studies have shown that the median time until abusive behavior is 24 months. Treatment targets and specific requirements are set to minimize this risk. \cite{marcus2009,stein2007}
The decision, if non-opioid or opioid analgesics are used, is based on weighing safety, tolerability and effectiveness. The superior effectiveness and the lower organ toxicity of opioid analgesic outweigh the risk of abuse or misuse. \cite{marcus2009} 
\end{itemize}

\subsection{Physical therapy}
Physical therapy is applied with the aim to enhance the patients' flexibility, general fitness and musculature. This is achieved by motion exercises and passive joint mobilization to enhance the muscle function and the joint stability and mobility. A special program is adapted to the patients' needs. Components of this program might be moist heat, cryo therapy, ultrasound and transcutaneous electrical stimulation. Furthermore, assistance can be provided by manual therapy or exercise, which is included to improve the physical fitness, achieve weight loss and decrease the risk of chronic diseases encouraged by inactivity. \cite{marcus2009,pope2017}

\subsection{Lifestyle changes}
Habits or life circumstances can intensify chronic pain. Changes of the lifestyle may help to decrease chronic pain. It is known that the pain sensitivity is negatively enhanced by nicotine. Therefore quit smoking can be a step towards relieving chronic pain.
Furthermore, chronic pain patients often suffer from insomnia. Sleep hygiene should be applied to reduce the occurrences as well as the severity of the sleep disturbances. If insomnia is due to medication, it should be revised, if it is possible to change the medication.
Obesity is a risk factor in the likelihood to  develop chronic pain, this is encouraged by the side effects of obesity like psychological disability or musculoskeletal pain. Besides, it encourages other health problems for example cardiovascular disease or diabetes. To improve this condition, weight loss should be achieved by the combination of diet and exercise. This will influence the recovery abilities from pain positively. \cite{marcus2009,pope2017}

\subsection{Psychological therapy}
%Psychological therapy helps patients to reduce depressions or anxiety and enhance a positive attitude. Also it assists patients to identify necessary lifestyle changes and implement them. \cite{marcus2009,pope2017} 

Psychological therapies have been promoted due to their potential effectiveness for the management of chronic pain and its consequences \cite{Eccleston2002}. Psychological treatments are characterized as cognitive or behavioral strategies \cite{Eccleston2013} which purpose is to helps patients to reduce depressions or anxiety and enhance a positive attitude. Also it assists patients to identify necessary lifestyle changes and implement them. \cite{marcus2009,pope2017}. The most popular treatment program is Cognitive Behavior Therapy (CBT). The patients learn that their chronic pain condition has no cure but can be managed using different skills such as relaxation training, environmental changes or behavioral experiments \cite{Burger2016}. It has been shown that CBT can reduce pain right after a treatment, however the effect is slight\cite{Eccleston2013}. Psychological trauma, victimization or serious emotional and relational conflict are not addressed on CBT, which is one of the limitations.
*** Citation is missing ***

\subsection{Surgery}
Surgery is a less frequent treatment technique. Commonly it is used to relieve patients from pain due to anatomic abnormalities. \cite{marcus2009,pope2017} In some cases surgery is not a recommended treatment, where risk and benefits of surgery should be considered. In addition to this is surgery also one of the most frequently cases of getting chronic pain, as mention in \autoref{sec:Pain}. Therefore, is should be weighted if it should be harked back to other and less invasive treatment options. \cite{pope2017}

\subsection{Alternative treatments}
Beside the primary treatments others can be considered in  addition to these, also known as alternative treatments. Most of the alternative treatments are often associated with spiritually and is not very well documented, why some people tend to be skeptical for these treatment types. However, many alternative treatments have shown to have an positive effect on chronic pain patients.

*** Citation is missing ***

\subsubsection{Acupuncture}
Acupuncture is a treatment method where small sterile needles are inserted into the skin of the patient. The needles are inserted at specific acupuncture points related to the type of pain that the patient is experiencing. \cite{Dhanani2011} Acupuncture has shown promising results in reducing pain in patients with soft tissue around the shoulder joint, headaches, neck and shoulder pain, arthritis/osteoarthritis and low back pain. The effect of acupuncture can last for more than 3 months in 80 \% of the patients. \cite{Junnilla1983} 
%A total of 348 patients where evaluated. The mean reduction of the entire patient group where 68 \%. Showing best results in soft tissue round the shoulder joint, showin a mean reduction of pain by 79 \%. The headache and neck and shoulder patients had a mean reduction by 74 \%. Patients with  arthritis/osteoarthritis showed a mean reduction by 58 \% and the patients with low back pain had a mean reduction by 50 \%. In 80 \% of the patients the effect of the treatment lasted for more than 3 month and 32 \% over one year. \cite{Junnilla1983}
% maybe one of these citeations are to old? 


\subsubsection{Chiropractor}
Chiropractic treatment is adjustment and manipulation of the spine in the patient to alignment the vertebrae of the spine to reduce pressure on the nerves running down the spine \cite{Gerald2013}. In some cases this therapy will after a few treatments, increase flexibility of the spine of the patient and relieve the pain \cite{Peterson2012}.
%In a study by \cite{Peterson2012} evaluating 506 patients with acute and chronic back pain after 3 month of chiropractic treatment.
%Patients undergoing chiropractic treatment showed improvements in their condition and the effect was ongoing after 3 month. \cite{Peterson2012}

\subsubsection{Hypnosis}
%Noxious stimuli increased cerebral blood flow in specific areas of the brain in pain processing,  a the thalamic nuclei anterior cingulate and insular cortices.
Hypnosis is a process where one comes into the state of trance and feels deep relaxation and is open to conversation verbally. Hypnosis is a guided process and can be carried out alone or by others. \cite{Gerald2013} Factors as anxiety, depression and other states of mood and in general the social life of the patient has been shown to play a role in chronic pain. These mechanisms might be altered by hypnosis.
In the literature hypnosis has shown positive results in pain relief, but only on a short term basis. \cite{Dhanani2011}
One of the drawbacks of hypnosis is the lack of standardization in hypnotic induction and interventions \cite{Alkins2010}. Also the hypnosis susceptibility  varies from person to person, why not everyone will have the same effect from it \cite{Spiegel2013}.

\subsubsection{Yoga}
Yoga is a form of mind to body practice discipline, or tradition originating from India. In the practice of yoga different physical postures, breathing techniques and more are the routine. 
Yoga is both, a form of personal evolution, but most popular because of the exercise which benefits the health. *** Citation is missing ***
A review by Whitehead et al. \cite{Whitehead2017} found that yoga could improve the functionality of the back and a slight effect of treating pain compared to non-yoga participants. 

\subsubsection{Meditation}
%*** Make it more meditation in general - end the section with something about mindfulness *** 

Meditation can be described as the intentional self-regulation of attention from moment to moment \cite{Kabat1982}.
The term meditation encompasses a variety of mental-training practice which depend on the mental activity promoted, the amount of training recommended, the use and qualifications of an instructor, and the degree of emphasis on religion or spirituality. There are some meditative techniques integrated into an alternative approach such as dietary or yoga. \cite{Goyal2014} Meditation practice can be divided into two main classes, concentration meditation and mindfulness or awareness meditation. While the concentration methods restrict the attention to a point, an object, the breath or a mantra (mental sound) \cite{Kabat1982}, mindfulness meditation is the practice of being aware in the present moment \cite{Zeidan2016}. The practice presupposes concentration to maintain steady attention instead of restricting attention to one object \cite{Kabat1982}. Mindfulness meditation practice is said to have several health benefits like increase in cognitive function and decrease in stress, depression and anxiety. Through some of these mechanisms pain can be altered, eventually leading in pain relief. \cite{Zeidan2016} 

%, a form of mental training. Mindfulness can be practiced through meditation, which is one of the common ways of practicing mindfulness. Mindfulness meditation practice is said to have several health benefits like increase in cognitive function and decrease in stress, depression and anxiety. Through some of these mechanisms pain can be altered, eventually leading in pain relief. \cite{Zeidan2016} 
