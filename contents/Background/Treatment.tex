\textbf{Treatment of chronic pain}

There are several ways of treatment for chronic pain patients, depending on the modalities and intensity of the pain.
Habits or life circumstances can intensify chronic pain. Changes of the lifestyle may help to decrease chronic pain. It is known that the pain sensitivity is negatively enhanced by nicotine. Therefore quit smoking can be a step towards relieving chronic pain.
Furthermore chronic pain patients often suffer from insomnia. Sleep hygiene should be applied to reduce the occurrences as well as the severity of the sleep disturbances. If insomnia is due to medication, it should be revised, if it is possible to change the medication to avoid medicine related insomnia.
Obesity is a risk factor in the likelihood to  develop chronic pain, besides, it encourages other health problems for example cardiovascular disease or diabetes. It is known that chronic pain occurs more often in people which are overweighted. This is encouraged by the side effects of obesity like psychological disability or musculoskeletal pain. To improve this condition, weight loss should be achieved by the combination of diet and exercises. This will influence the recovery abilities from pain positively.

Another way to treat chronic pain is by physical therapy with the aim to enhance the patients' flexibility, general fitness and musculature. A special program is adapted to the patients' needs. Components of this program might be moist heat, cryo therapy, ultrasound and transcutaneous electrical stimulation. Furthermore assistance can be provided by manual therapy or exercise, which should be included to improve the physical fitness, achieve weight loss and decrease the risk of chronic diseases encouraged by inactivity.

Psychological therapy helps patients to reduce depressions or anxiety. Also it assists patients to identify necessary lifestyle changes and implement them.

Surgery is a less frequent treatment technique. Commonly it is used to relieve patients from pain due to anatomic abnormalities.

Medication is a common way to treat severe chronic pain patients, although there are disadvantages. Those medicaments can be divided in three groups, the coanalgesic medicaments, the non-opioid and the opioid analgesics. 
Coanalgesics are used to treat other diseases, but provide analgesic qualities. They are often used to treat fibromyalgia, chronic headache and neuropathic pain. Often coanalgesics are combined with analgesicts to extended pain-relief.
Non-opioid analgesics are used to reduce intermittent mild to moderate pain. To this category belong nonsteroidal anti-inflammatory drugs, which decrease inflammation and give analgesic properties. Non-opioid analgesics are especially used in short-term-therapy. A permanent use of Non-opioid analgesics encourages prostaglandin effects, which conduct in severe organ toxicity. Known side effects are for example gatrointestinal toxicity, pephrotoxicity and a increased risk of cardiovascular diseases.
Opioid analgesics provide stronger analgesic qualities than non-opioid analgesics and show no prostaglandin effect. Because of this better long-term tolerability opioid analgesics are used in patients which suffer from chronic non-malignant pain. But the use of opioid analgesics accompanies with the risk of abuse and misuse. Studies have shown, that the median time until abuse behavior is 24 months. Treatment targets and specific requirements are set to minimize this risk.
The decision, if non-opioid or opioid analgesics are used, is based on weighing safety, tolerability and effectiveness. The superior effectiveness and the lower organ toxicity of opioid analgesic outweigh the risk of abuse or misuse.

None of the different treatment methods is enough or sufficient when applied alone. But a individual combination considering the needs of each patient alleviates the suffering of the chronic pain.