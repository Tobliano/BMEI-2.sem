\section{State of the Art} \label{sec:SOTA}
%Chronic pain have been investigated for years in order to understand the mechanism behind. Even though, is it still a relevant issue to explore as many people suffer from chronic pain. Furthermore, is it difficult to treat due to that pain is experienced individually. This also leads to an issue as the assessment of pain is subjective.

%Currently, is there no cure for chronic pain only relieving treatments. The primary treatment is pharmaceutical, which may have side effects, such as toxicity. Alternative treatments, such as therapy, chiropractic and acupuncture have shown to have an impact on relieving pain, but only in a combination with pharmaceutical treatments. The alternative treatments have some disadvantaged due to high costs, why these treatment should be considered.  

%Mindfulness meditation have proved to relieve conditions such as stress, depression and anxiety due to meditations ability to enhance emotion regulation, cognitive control, acceptance and positive mood \cite{Zeidan2012, Zeidan2016}. Studies have also investigated the usefulness of mindfulness meditation for people with chronic pain with similar results.

%Previous studies… 

Chronic pain has been investigated for years in order to understand the mechanisms behind and the topic is still relevant to explore as many people suffer from chronic pain. Furthermore, it is an issue that pain is difficult to treat due to the individually experience of pain and the subjective assessment of pain. \cite{Briggs2010,Norton1999}

Currently there is no cure for chronic pain, only relief treatments. The primary treatment is pharmaceutical, which has possible side effects, such as abuse or toxicity. Alternative treatments like physical therapy, chiropractic and acupuncture have shown an impact in relieving pain. But these treatments will most likely be used in combination with pharmaceutical treatments. Many alternative treatments have disadvantages such as high costs, why the decision for these treatments should be considered well, to ensure that it suits the patients’ needs and to maximize the effect for the cost. \cite{marcus2009,pope2017}

Mindfulness meditation has proved to relieve conditions such as stress, depression and anxiety through the ability to enhance emotion regulation, cognitive control, acceptance and positive mood \cite{Zeidan2012,Zeidan2016}. Studies have investigated the usefulness of mindfulness meditation for people with chronic pain showing promising results in pain relief. \cite{Kabat1982,Rosenzweig2010}

%Mindfulness meditation was conceived in the late 1970s and spotted for patients suffering from chronic pain \cite{Chiesa2010}. There is ambiguity within the literature due to the different meditative practices under the term 'mindfulness'. Different studies divide this technique in two styles FA and OM \cite{Lutz2008,Zeidan2012}. Some surveys presented that particularly OM practice is associated with pain reduction \cite{Grant2009,Perlman2010}. 
% **** ADD something about FA - that it maybe easier?? **

The most commonly used mindfulness-based intervention is Mindfulness Based Stress Reduction (MBSR) \cite{Cramer2012}. MBSR consist of 8 or 10 weeks of mindfulness meditation where the patient have to attend once a week a course for 2 hours and 45 minutes session at home 6 days per week \cite{Kabat1982, Chiesa2010}. Patients suffering from chronic pain improved pain symptoms as well as life quality after the finalization of MBSR \cite{Zeidan2012}.
%An extent of MBSR is Cognitive Behavioral Therapy (CBT)\fxnote{Check CBT!}, which incorporates elements of cognitive therapy facilitating a detached view of one's thoughts and is designed to prevent depressive relapse \cite{Chiesa2010}. 
Patients suffering from chronic pain improved pain symptoms as well as life quality after the finalization of MBSR. \cite{Zeidan2012} 
%
%However, studies showed no difference using MBSR or CBT to relieve chronic pain, as both illustrated positive effect compared with usual care \cite{Jacob2016,Cherkin2016}. 
Hence, MBSR provides significant improvement for patients suffering from neck pain \cite{Rosenzweig2010}.

Even though long-term mindfulness meditation is the most investigated, short-term mindfulness training has also shown relief of pain. The studies have investigated different duration of meditation practice and time period within short-term mindfulness meditation. Consequently the boundaries of short-term mindfulness meditation are not well defined. A study by Ussher et.al. \cite{Ussher2012} showed in a clinical setting that 10 minutes mindfulness-based body scan reduces distress and the perception of pains’ impact on daily living. However, the study found no effect outside the clinical environment \cite{Ussher2012}. Another study by Zeidan et.al. \cite{Zeidan2012} proved that only three days of mindfulness meditation with a 20 minutes session each day have an effect on relieving chronic pain. 

Some studies have studied the effect of mindfulness meditation for musculoskeletal chronic pain unifying lower and upper back pain, shoulder and cervical pain \cite{Chiesa2010}. Nevertheless, there is not much literature available focusing on chronic neck pain, which about 25 \% of the patients suffer from \cite{Macfarlanea2016}. The most investigated method is MBSR, mostly over a time period of two months or more. A shorter time period of mindfulness meditation has not been investigated focusing on neck pain.

