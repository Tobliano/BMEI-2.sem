\chapter{Introduction}

Probably everybody experienced pain once, either due to a cut, a burn or a fall. The pain occurring right after an injury is called acute pain and disappears near-term.\fxnote{ref}

Approximately 20\% of the population suffer permanent from pain, so-called chronic pain\fxnote{ref}. The characteristics of chronic pain is a duration more than three months. Due to the persistence of pain the patients get restricted physically as well as psychically. %Sinkende Lebensqualität, ggf. Arbeitslosigkeit und Depression...

27 / 15 \% of the chronic pain patients suffer from low back / neck pain \fxnote{ref}. Those patients are restricted by ...

At the moment there is no cure for those chronic pain patients. The current treatment methods only provide possibilities to relieve the pain. Nevertheless the patients feel pain daily.
Chronic pain is mainly treated by medication. However those medicaments have side effects like abuse or organ damage. To avoid those risks, alternative methods are used. One of those methods is mindfulness meditation. %Hierbei...

Previous studies show, that ...

The present study addressed the question if mindfulness meditation can relieve low back / neck pain. Therefore an experiment was conducted whereby the pressure pain threshold and the pressure pain tolerance of the subjects were measured by an algometer. After five days with 20 min mindfulness meditation each the pressure pain threshold and the pressure pain tolerance was measured again. To ensure that the difference between the measurements is not caused by adaption or habituation to the pressure a controlled trial has been performed and a control group has been measured with the same time difference but without mindfulness meditation.
This experiment was conducted to either accept or reject the hypothesis \textit{"Short-term mindfulness meditation increases the pressure pain threshold and the pressure pain tolerance."}.