\chapter{Introduction}

Probably everybody experienced pain once, either due to a cut, a burn or a fall. The pain occurring right after an injury is called acute pain and disappears near-term. \cite{Briggs2010;Mello2016}

Approximately 1.5 billion of people \cite{Zeidan2016}, which equals 20\% of the population suffer permanent from pain, so-called chronic pain \cite{Macfarlanea2016}. The characteristics of chronic pain is a duration more than three months \cite{Mello2016}. Due to the persistence of pain the patients get restricted physically as well as psychically. 
The patients' ability to participate in diverse activities decreases. Those activities are not only exercising, walking or lifting, but also social activities. So are also maintaining an independent lifestyle ans relationships to friends as well as to family, sexuality and sleeping affected. Besides the impacts on life, pain has impact on the work life. 25\% indicated in a survey that the persistence of pain had a lasting effect on their employment status. These patients changed their job, the job responsibilities or lost the job. 
As a result of this impacts 21\% of the chronic pain patients are diagnosed with depression. \cite{Breivik2006}

15 \% of the chronic pain patients suffer from neck pain \cite{Macfarlanea2016}. Those patients are restricted by negatively affected fatigue and concentration \cite{vanRanderaat2016}. Furthermore they suffer like all chronic pain patients from anxiety and depressed mood, cognitive distress and the resulting physical limitations. \cite{gross2013}

At the moment there is no cure for those chronic pain patients. The current treatment methods only provide possibilities to relieve the pain. \cite{marcus2009;pope2017} Nevertheless the majority of the patients feel pain daily and this pain is increasing throughout the day due to the daily activities. \cite{Breivik2006}
Chronic pain is mainly treated by medication. However those medicaments have side effects like abuse or organ damage. To avoid those risks, alternative methods are used. One of those methods is mindfulness meditation. Whereby meditation is used as mental training to achieve diminished judgment of emotions, cognitive control and existential insight. \cite{zeidan2012}

Previous studies show that mindfulness meditation provides the ability to enhance a broad spectrum of cognitive health outcomes. Furthermore, stress, depression and anxiety can be enhanced. This improvements are due to the mental training achieved by mindfulness meditation. Especially because of emotion regulation, cognitive control, acceptance and positive mood. \cite{Zeidan2012;Zeidan2016} ...

The present study addressed the question if mindfulness meditation can relieve low back / neck pain. Therefore an experiment was conducted whereby the pressure pain threshold and the pressure pain tolerance of the subjects were measured by an algometer. After five days with 20 min mindfulness meditation each the pressure pain threshold and the pressure pain tolerance was measured again. To ensure that the difference between the measurements is not caused by adaption or habituation to the pressure a controlled trial has been performed and a control group has been measured with the same time difference but without mindfulness meditation.
This experiment was conducted to either accept or reject the hypothesis \textit{"Short-term mindfulness meditation increases the pressure pain threshold and the pressure pain tolerance."}.