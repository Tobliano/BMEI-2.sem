\section{Treatment}
\begin{longtable} {l|c|c|c|c|c|c}
 \rowcolor[HTML]{C0C0C0} 
  \color[HTML]{000000}{} & 
 \multicolumn{3}{c|}{ \color[HTML]{000000}{\textbf{Threshold}}} & \multicolumn{3}{c}{ \color[HTML]{000000}{\textbf{Tolerance}}}  	\\  \rule{0pt}{3ex} 
  \cellcolor[HTML]{C0C0C0}{} &
 \multicolumn{1}{c|}{ \cellcolor[HTML]{C0C0C0}{Pre [KgF]}} & \multicolumn{1}{c|}{ \cellcolor[HTML]{C0C0C0}{Post [KgF]}} 
 & \multicolumn{1}{c}{ \cellcolor[HTML]{C0C0C0}{\textcolor[HTML]{C0C0C0}{0}Diff [\%]\textcolor[HTML]{C0C0C0}{0}}}
 & \multicolumn{1}{|c|}{ \cellcolor[HTML]{C0C0C0}{Pre [KgF]}} 
 & \multicolumn{1}{c|}{ \cellcolor[HTML]{C0C0C0}{Post [KgF]}} 
 & \multicolumn{1}{c}{ \cellcolor[HTML]{C0C0C0}{\textcolor[HTML]{C0C0C0}{0}Diff [\%]\textcolor[HTML]{C0C0C0}{0}}}  	\\ \hline 
\#T1 & 1.84 & 1.53 & -17.03 & 4.21 & 3.93 & -6.66 \\ \hline
\#T2 & 2.95 & 2.85  & -3.50 & 7.63  & 5.70 & -25.26 \\ \hline
\#T3 & 2.13 & 3.07 & 43.75 & 7.75 & 8.13 & 4.99 \\ \hline
\#T4 & 0.94 & 2.34  & 148.94  & 3.85 & 4.95 & 28.55 \\ \hline
\#T5 & 1.35 & 1.71  & 26.11 & 3.11 & 3.94 & 26.55 \\ \hline	
\#T6 & 0.31 & 0.94   & 206.52 & 5.95 & 5.99 & 0.67 \\ \hline
\#T7 & 2.07 & 2.74  & 32.15 & 5.44 & 8.82 & 62.13 \\ \hline
\#T8 & 1.82 & 3.59 & 97.44 & 7.21 & 10.11 & 40.11 \\ \hline
\#T9 & 2.17 & 2.84  & 31.08 & 6.98 & 9.62 & 37.82 \\ \hline
\#T10 & 4.71 & 4.85  & 3.12 & 12.37*  & 13.24* & 7.06 \\ \hline
\#T11 & 2.22 & 4.31 & 93.99 & 4.45 & 7.76 & 74.25 \\ \hline
\#T12 & 1.99 & 2.51 & 26.51 & 4.45 & 4.79 & 7.49 \\ \hline
\#T13 & 1.14 & 2.37 & 108.19 & 4.48 & 6.57 & 46.58 \\ \hline
\#T14 & 1.69 & 1.01 & -40.55 & 6.04 & 3.93 & -34.88 \\ \hline
\#T15 & 2.03 & 2.58 & 27.30 & 8.57* & 14.28* & 66.56 \\ \hline
\#T16 & 2.79 & 2.39 & -14.32 & 13.35* & 13.59* & 1.80 \\ \hline
\#T17 & 3.24 & 2.76 & -14.81 & 11.75 & 11.77 & 0.11 \\ \hline
\#T18 & 2.16 & 1.98 & -8.33 & 8.38* & 11.93* & 42.32 \\ \hline
\#T19 & 1.77 &  & &9.66 & &  \\ \hline
\#T20 & 2.28 &  & & 7.20     & & \\ \hline
\#T21 & 3.91 &  & & 7.18 & & \\ \hline
	\caption{Measured threshold before and after and tolerance before and after for the treatment group. Furthermore, is the percentage difference for both threshold and tolerance illustrated. The asterisk indicated that the tolerance was not representative.}
	\label{tab:Treatment}
\end{longtable}
\vspace{-.5cm}

\section{Control}
\begin{longtable} {l|c|c|c|c|c|c}
 \rowcolor[HTML]{C0C0C0} 
  \color[HTML]{000000}{} & 
 \multicolumn{3}{c|}{ \color[HTML]{000000}{\textbf{Threshold}}} & \multicolumn{3}{c}{ \color[HTML]{000000}{\textbf{Tolerance}}}  	\\  \rule{0pt}{3ex} 
  \cellcolor[HTML]{C0C0C0}{} &
 \multicolumn{1}{c|}{ \cellcolor[HTML]{C0C0C0}{Pre [KgF]}} & \multicolumn{1}{c|}{ \cellcolor[HTML]{C0C0C0}{Post [KgF]}} 
 & \multicolumn{1}{c}{ \cellcolor[HTML]{C0C0C0}{\textcolor[HTML]{C0C0C0}{0}Diff [\%]\textcolor[HTML]{C0C0C0}{0}}}
 & \multicolumn{1}{|c|}{ \cellcolor[HTML]{C0C0C0}{Pre [KgF]}} 
 & \multicolumn{1}{c|}{ \cellcolor[HTML]{C0C0C0}{Post [KgF]}} 
 & \multicolumn{1}{c}{ \cellcolor[HTML]{C0C0C0}{\textcolor[HTML]{C0C0C0}{0}Diff [\%]\textcolor[HTML]{C0C0C0}{0}}}  	\\ \hline   
\#C1 & 3.04	& 5.00	&	64.47	& 7.80	& 	12.07 &	54.79\\ \hline
\#C2 & 1.85 	& 2.27	&	22.30	& 7.35	& 	9.45 & 28.68	\\ \hline
\#C3 & 1.92 	& 1.81	&	-5.90	& 4.90	& 	4.32 & -11.84	\\ \hline
\#C4 & 1.93 	& 2.09	&	7,93		& 6.25	&	7.31 & 16.97	\\ \hline
\#C5 & 2.01 	& 4.73 	& 	134.77	& 11.46* 	& 13.77* & 20.19		\\ \hline
\#C6 & 2.60 	& 3.45	& 	32.56		& 7.85*	& 14.05 & 79.01		\\ \hline	
\#C7 & 3.60 & 4.34	& 	20.56		& 6.57 & 8.96  &	36.31 \\ \hline
\#C8 & 1.98 & 2.57	& 	29.97		& 10.25	& 10.91 &	6.37	\\ \hline
\#C9 & 2.59 & 3.19 	& 	7.9		& 8.89	& 9.51 & 6.90		\\ \hline
\#C10 & 4.61 & 6.80	& 	47.61		& 12.85*	& 10.65* & -17.17 \\ \hline
\#C11 & 1.27 & 1.29 	& 	1.58		& 3.56	& 5.21 &  46.25\\ \hline
\#C12 & 2.31 & 4.32 	& 	87.28	& 9.45 & 10.05 & 6.28 \\ \hline
\#C13 & 4.47 & 2.56 	& 	-42.69	& 8.51 & 9.67 & 13.67 \\ \hline
\#C14 & 1.85 & 3.07 & 	66.43	 & 5.17 & 7.00 & 35.31 \\ \hline
\#C15 & 1.14 & 1.98 & 	73.68 & 5.83 & 5.17 & -11.43 \\ \hline
\#C16 & 2.05 & 2.06 & 	0.32 & 8.21 & 7.98 & -2.84 \\ \hline
\#C17 & 1.52 & 1.81 &	18.86 & 10.77 & 6.91 & -35.79 \\ \hline
\#C18 & 1.98 & 2.05 & 	3.70 & 4.26  &  4.36 & 2.35 \\ \hline
\#C19 & 5.58 & 2.36 & 	-39.78 & 18.17* & 14.39* & -20.81 \\ \hline
\#C20 & 2.41 & 2.97 &  23.27 & 7.80 &  8.96 & 14.87\\ \hline
\#C21 & 3.83 & 4.06 & 5.91  & 11.65 & 11.32 & -2.78 \\ \hline
	\caption{Measured threshold before and after and tolerance before and after for the control group. Furthermore, is the percentage difference illustrated for both threshold and tolerance. The asterisk indicated that the tolerance was not representative.}
	\label{tab:Control}
\end{longtable}



\section{Total}
\begin{longtable} {l|c|c}
 \rowcolor[HTML]{C0C0C0} 
  \color[HTML]{000000}{} & 
\color[HTML]{000000}{\textbf{Threshold}} & \color[HTML]{000000}{\textbf{Tolerance}} 	\\  \rule{0pt}{3ex} 
\cellcolor[HTML]{C0C0C0}{} &
 \cellcolor[HTML]{C0C0C0}{Difference [\%]} &  \cellcolor[HTML]{C0C0C0}{Difference [\%]} \\ \hline
Treatment &  &  \\ \hline
Control & 25.34 & 12.63  \\ \hline
	\caption{Total}
	\label{tab:Total}
\end{longtable}
\vspace{-.5cm}


\section{Statistics}
As the sample size is small the Shapiro-Wilk is used for normality test. The p-value should be above 0.05 to reject the null hypothesis. This is for ANOVA...

\begin{longtable} {l|c|c|c|c}
 \rowcolor[HTML]{C0C0C0} 
   \color[HTML]{000000}{} & 
  \color[HTML]{000000}{\textbf{Threshold Pre}} & 
\color[HTML]{000000}{\textbf{Threshold Post}} &
\color[HTML]{000000}{\textbf{Tolerance Pre}}  &
\color[HTML]{000000}{\textbf{Tolerance Post}}
 \\ \hline
Treatment(21)&  &  & & \\ \hline
Control(21)& 0.016  & 0.080* & 0.514*  & 0.155* \\ \hline
Treatment(17)&  &  & & \\ \hline
Control(17)& & & & \\ \hline
	\caption{Shapiro-Wilk Test for Normality for Threshold and Tolerance pre and post for treatment and control respectively. The asterisk indicate normality}
	\label{tab:ShapiroWilk1}
\end{longtable}
\vspace{-.5cm}

As all data were not normally distributed a Kruskall wallis test was tested.
\begin{longtable} {l|c|c|c|c}
 \rowcolor[HTML]{C0C0C0}  \color[HTML]{000000}{} & 
   \color[HTML]{000000}{\textbf{Treshold Pre}} & 
  \color[HTML]{000000}{\textbf{Threshold Post}} & 
   \color[HTML]{000000}{\textbf{Tolerance Pre}} & 
\color[HTML]{000000}{\textbf{Tolerance Post}}  
 \\ \hline
All subject (21) &  &  &   & \\ \hline
Not all subject (17) &  &  &   & \\ \hline
	\caption{Kruskal Wallis Test for threshold and tolerance difference for treatment and control respectively. The asterisk indicate normality}
	\label{tab:KruskalWallis}
\end{longtable}
\vspace{-.5cm}

If the p-values is below 0.05 make a post hoc test to see where the different are.




As the sample size is small the Shapiro-Wilk is used for normality test. The p-value should be above 0.05 to reject the null hypothesis. This is for T-test...

\begin{longtable} {l|c|c}
 \rowcolor[HTML]{C0C0C0} 
   \color[HTML]{000000}{} & 
  \color[HTML]{000000}{\textbf{Threshold  Difference}} & 
\color[HTML]{000000}{\textbf{Tolerance Difference}}  
 \\ \hline
Treatment(21) &  &   \\ \hline
Control(21) & 0.071*  & 0.393*  \\ \hline
Treatment(17) &  &   \\ \hline
Control(17) & & \\ \hline
	\caption{Shapiro-Wilk Test for Normality for threshold and tolerance difference for treatment and control respectively. The asterisk indicate normality}
	\label{tab:ShapiroWilk2}
\end{longtable}
\vspace{-.5cm}

Normal = T-test Not normal = Mann Whitney ..
\begin{longtable} {l|c|c|c|c}
 \rowcolor[HTML]{C0C0C0}  \color[HTML]{000000}{} & 
   \color[HTML]{000000}{\textbf{Treshold Pre}} & 
  \color[HTML]{000000}{\textbf{Threshold Post}} & 
   \color[HTML]{000000}{\textbf{Tolerance Pre}} & 
\color[HTML]{000000}{\textbf{Tolerance Post}}  
 \\ \hline
All subject (21) &  &  &   & \\ \hline
Not all subject (17) &  &  &   & \\ \hline
	\caption{Kruskal Wallis Test for threshold and tolerance difference for treatment and control respectively. The asterisk indicate normality}
	\label{tab:KruskalWallis}
\end{longtable}
\vspace{-.5cm}

