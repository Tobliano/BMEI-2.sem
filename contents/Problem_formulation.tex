\chapter{Problem formulation}
%Approximately 375 million people suffer from chronic neck pain. The primary treatment for those patients is medication. But medication has side effects, as described in \ref{sec:treatment}.  Besides medication, alternative treatment methods are used, often in combination with medication. For example physical therapy, chiropractor or psychological therapy have showed a positive influence on pain relief. Most of the alternative treatment methods are related with high costs, because they require a specialist for the application. Whereas mindfulness meditation can be practiced alone. Hence a lot of studies focused on the ability of mindfulness meditation to relieve pain.
%As mentioned in \ref{sec:SOTA}, there are not many studies which show the effect of mindfulness meditation on chronic neck pain. Since a lot of people suffer from chronic neck pain this study investigates the influence of mindfulness meditation on neck pain. 

Pain levels of chronic pain patients are difficult to assess and quantify. Chronic pain patients experience a habituation effect to the pain. Additionally there are variations of the pain sensitivity between days and throughout the day which makes it difficult to get reliable and comparable pain levels of these patients. Since chronic pain is a subjective and multidimensional experience, another way to get comparable values of pain was chosen. Therefore pressure pain was applied with an algometer on healthy subjects. The application of pressure with an algometer was chosen to assess pain, because pressure satisfies the requirements of an suitable method for the quantification of pain \cite{Keele1954}. Hence pressure pain threshold (Threshold) and pressure pain tolerance (Tolerance) values were used to test the following hypothesis:
%\textit{Short-term mindfulness meditation practice increases the pressure pain threshold and the pressure pain tolerance in the upper trapezius.}
\textit{Short-term mindfulness focused attention meditation practice on 5 consecutive days increases the pressure pain threshold and pressure pain tolerance in the right upper trapezius}. In this study meditation was used, because physical limitations are not affecting the meditation practice and no prior knowledge is needed \cite{Tang2017}. In mindfulness meditation FA is used to slide into OM, therefore it is necessary to master FA before one can reach OM. \cite{Perlman2016, Zeidan2016,Kabat1982} Hence the chosen meditation practice in this study is FA. The upper trapezius was chosen for the pressure application, because it is involved in chronic neck pain and has the lowest Threshold values compared with other muscles \cite{Falla2004,Fischer1987}. Even though the study was conducted in healthy subjects, the sensation of the pain and the effects of meditation to the pain sensitivity can be transferred to chronic pain patients.


